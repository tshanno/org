
% Created 2017-11-13 Mon 14:23
\documentclass[11pt]{article}
\usepackage[utf8]{inputenc}
\usepackage[T1]{fontenc}
\usepackage{fixltx2e}
\usepackage{graphicx}
\usepackage{longtable}
\usepackage{float}
\usepackage{wrapfig}
\usepackage{rotating}
\usepackage[normalem]{ulem}
\usepackage{amsmath}
\usepackage{textcomp}
\usepackage{marvosym}
\usepackage{wasysym}
\usepackage{amssymb}
\usepackage{booktabs}
\usepackage[colorlinks = true,
            linkcolor = blue,
            urlcolor  = blue,
            citecolor = blue,
            anchorcolor = blue]{hyperref}
\tolerance=1000

\renewcommand{\familydefault}{cmss}

\addtolength{\oddsidemargin}{-.875in}
	\addtolength{\evensidemargin}{-.875in}
	\addtolength{\textwidth}{1.75in}

	\addtolength{\topmargin}{-.875in}
	\addtolength{\textheight}{1.75in}


\begin{document}
The authors address the effects of loss of MD1 increased upon susceptibility to ventricular arrhythmias in obesity mice.  They measure K and Ca current densities and kinetics and action potential duration in mice fed a high fat diet (MD1-KO Vs. WT).  They also measure mRNA levels for each K channel. They hypothesize that loss of MD1 reduces K+ currents thus prolonging the action potential and increasing this susceptibility.  They further hypothesize that there is a reduction in L-type Ca current in these myocytes.  The data support the hypothesis.

Major Issues

The data look fine and the reviewer accepts that MD1, along with many other proteins, must play a role in normal cardiac function.  But the authors fail to describe even briefly what MD1 is and why they are studying it in relation to cardiac disease.  Though the data indicate that loss of the protein results in cardiac dysfunction when coupled with a high fat diet they fail to describe an actual connection to disease.  Is it down-regulated in diseased mice?  Does the level of MD1 change with obesity if it isn't knocked out, thus potentially causing or exacerbating the effect in normal mice?  The reviewer went to considerably more trouble than the average reader will to search and find these answers and came up empty.

It is evident that the loss of MD1 causes a problem.  But based upon the text of the manuscript, there are no conditions under which any sort of loss would be seen in a physiological context.  The authors need to better defend the significance of this project by drawing a direct, mechanistic connection to a pathophysiological problem.

\end{document}
