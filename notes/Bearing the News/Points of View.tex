% Created 2019-02-17 Sun 09:59
% Intended LaTeX compiler: pdflatex
\documentclass[11pt]{article}
\usepackage[utf8]{inputenc}
\usepackage[T1]{fontenc}
\usepackage{graphicx}
\usepackage{grffile}
\usepackage{longtable}
\usepackage{wrapfig}
\usepackage{rotating}
\usepackage[normalem]{ulem}
\usepackage{amsmath}
\usepackage{textcomp}
\usepackage{amssymb}
\usepackage{capt-of}
\usepackage{hyperref}
\author{Tom Shannon}
\date{\today}
\title{}
\hypersetup{
 pdfauthor={Tom Shannon},
 pdftitle={},
 pdfkeywords={},
 pdfsubject={},
 pdfcreator={Emacs 26.1 (Org mode 9.1.9)}, 
 pdflang={English}}
\begin{document}

\subsection{Bears}
\begin{itemize}
\item  \textbf{Kevin Fishbain} at \emph{The Athletic} \href{https://theathletic.com/990751/2019/05/22/ota-takeaways-the-bears-arent-hibernating-as-the-defending-nfc-north-champs-show-up-at-halas-hall/}{addresses the situation of \textbf{Taquan Mizzell}}, last years deep back up at running back:

\begin{quote}
``Mizzell probably received the most criticism for a player not named \textbf{Cody Parkey} last season, which is impressive for someone who had only nine carries and eight receptions on 69 snaps. 

``He’ll have a lot of work to do to make the team in a crowded receivers room. This move clears up the running back depth chart a bit. It’s now \textbf{Kerrith Whyte Jr.} vs. \textbf{Ryan Nall} for the No. 4 spot.''
\end{quote}

Count me among those critical of the Bears use of Mizzell in any situation last year.  Sixty nine snaps is plenty enough to get on people's radar and, like many observers, I couldn't figure out why he was on the field at all.  It was evident to me that he just wasn't that good.

You wonder why the Bears don’t just release Mizzell. There must be something about him that someone likes, probably head coach \textbf{Matt Nagy}. He’s going to be a practice squad project at wide receiver and you wonder if that spot would be better used on someone else and if Mizzell, himself, would be better off going elsewhere without the position change where he’ll have a better chance to play.

From what I've seen the odds are slim that he'll ever develop into the kind of player that could crack the starting lineup with the Bears.

\item Fishbain also quotes defensive backs coach \textbf{Deshea Townsend} \href{https://theathletic.com/990751/2019/05/22/ota-takeaways-the-bears-arent-hibernating-as-the-defending-nfc-north-champs-show-up-at-halas-hall/}{on the importance of tackling in the secondary}. 

\begin{quote}
``'The one thing that we can’t do every day is tackle,' [Townsend] said. 'A lot of people get the misconception of DBs doing drills and it’s always footwork, but we’re going to find some way to wrap and squeeze every day.  (I’m) always talking about angles. This whole game is angles. But we’re gonna do a tackle drill every day. And it’s just going to reinforce what we are. Even saying we’re going to be the best tackling secondary in the NFL, that’s our goal. If you’re not saying that, if you don’t believe it, it’s not gonna happen. That has to be the mindset of everybody in the group.'''
\end{quote}

Fishbain is a former college defensive back so he knows the importance of this aspect of playing in the secondary even if others don't appreciate it.

It’s nice that this is on Townsend’s mind. In the few bad games the Bears defense had last year the tackling was horrendous. See \href{http://bearingthenews.com/blog/2018/10/14/quick-game-comments-bears-dolphins-101418/}{my comments on the Miami loss after the bye week} as a good example. Avoiding those let downs will be one of the keys to improvement this year.

\textbf{Adam Jahns}, also at \emph{The Athletic} gives us \href{https://theathletic.com/991989/2019/05/23/jahns-dont-forget-about-roquan-smith-the-bears-elite-lb-in-the-making/}{the obligatory hype about Bears players that you get in May}.  In this case its Bears second year player \textbf{Roquan Smith}.

\begin{quote}
``'I feel like I can improve in a lot of areas,' Smith said.

``If we’re nitpicking, Smith needs to improve in coverage, but that should come through his own experiences, including in practices against running backs \textbf{Tarik Cohen} and \textbf{David Montgomery}. There is a significant difference between seeing an option route from a back on film and handling one at game speed.

``When [inside linebackers coach \textbf{Mark}] \textbf{DeLeone} evaluated Smith’s rookie film, he saw a young linebacker who started to shake off the 'rust' and improve as the weeks went by.''
\end{quote}

That's not nitpicking.  It was a serious problem and it wasn't just Smith.  The Bears were constantly getting burned last season as receivers dragged across the middle of the field where the inside linebackers failed to pick them up in coverage.

In fairness, it doesn't look like it's particularly easy to do.  I would imagine that it's tough enough to be completely aware of what's going on in front of you let alone of opposing players coming at you on routes from the side and slightly behind you.  Nevertheless, that's what these guys get paid to do.

Not to beat a dead horse but as DeLeone points out, you have to feel that Smith would have done a better job of getting on top of this had his agent not held him out.  Here's hoping that with a full offseason Smith, along with the other linebackers, does a better job of correcting this issue.

\end{itemize}

\subsection{Elsewhere}
\begin{itemize}
\item \textbf{Dave Birkett} at the \emph{Detroit Free Press} \href{https://twitter.com/davebirkett/status/1130120949548949504?s=12}{checks in with the latest odds on who will be on HBO's Hard Knocks program}.

\begin{tabular}{ll}
  \multicolumn{2}{l}{\textbf{Hard Knocks 2019 - Team Featured}
  Washington Redskins &5/4\\
  Oakland Raiders &5/2\\
  New York Giants &3/1\\
  Detroit Lions &7/2\\
  San Francisco 49ers &9/1\\ 
\end{tabular}

I understand why \textbf{Daniel Snyder}'s Redskins might be the favorite.  Snyder seems like just the entrepreneur who would see this as an opportunity rather than a detriment.  Nevertheless my money's on the Raiders.

\textbf{Mark Davis} has been adamantly against this team appearing in the past.  But getting permission to move his franchise to Vegas undoubtedly came with a lot of strings attached behind the scenes.  The bet here is that it's not coincidence that the Rams both appeared on Hard Knocks and went to London to play after permission to move to Line of scrimmage Angeles was given.

Oakland plays a home game against the Bears in London this year and it would surprise no one if they ended up being forced to volunteer to be on Hard Knocks as well.

\item \textbf{Mike Florio} at \textit{profootballtalk.com} \href{https://profootballtalk.nbcsports.com/2019/05/19/would-jets-trade-leveon-bell/}{speculates upon the possibility that Jets head coach \textbf{Adam Gase} may be inclined to trade \textbf{Le'Veon Bell}} after rumors surfaced that he never wanted to sign the back.

\begin{quote}
``If Gase is inclined to do it, now’s the time given his current power and control over the team. And \textbf{John Clayton}, formerly of ESPN and now a radio host in Seattle, recently said just enough on 93.7 The Fan in Pittsburgh to get people thinking that a trade could happen: 'If there’s a suitor, I could absolutely see the Jets trading him before the start of the season.'''
\end{quote}

The thought that the Jets would trade Bell is ridiculous.  The Jets have already paid Bell \$12 million and \href{https://overthecap.com/player/leveon-bell/2258/}{the cap hit if they traded him would be extremely high}.

Furthermore I have a hard time believing that anyone wouldn't want Le'Veon Bell, let alone an offensive coach like Gase.  I have little trouble believing that Gase thought they overpaid.  But the thought that Gase would trade him now that they've paid Bell is absurd.

Personally I never felt that John Clayton added much in the way of reporting when he was with ESPN.  And I definitely don’t think he knows what he’s talking about now.

\end{itemize}


\subsection{One Final Thought}
\textbf{Albert Breer} at \emph{SI.com} \href{https://www.si.com/nfl/2019/05/20/colts-2019-offseason-chris-ballard-chris-long-retirement-patrick-peterson-suspension-mmqb}{writes about how the Colts are gradually shrinking their draft board year to year}.

\begin{quote}
``I’d say this year we had 170 players on the board [for 2019], which is way down from where it was before,' [General Manager \textbf{Chris}] \textbf{Ballard} said. 'I think last year we were at 220, I can’t even remember the number from my first year. But yeah, it makes it easier to navigate when you have fewer names that you know fit what you want. I think when we really get it right, and we get it down to about 125, 150, that’s when we’ll have really honed down exactly what a Colt is for our schemes.'''
\end{quote}

What the Colts are doing is a lesson for us all.  I have found that being brutal about cutting things out of my life, from tossing things from storage to pruning task lists, makes it a lot easier to get better results in the end.

Honestly, if you have something in your closet that you haven't touched for five years, are you really going to need it in the next 5?  Or the 5 after that?

Anyway, this is a sign that the Colts really know what they are doing.  The bet here is that going into the draft, any general manager worth his salt probably knows deep in his heart that there are only 50 or so players they are really likely they'll end up with.  Maybe even less.  So why put 350 on your board?

The ability to hone in on what's really important and trimming the rest seems to be one underrated key to success.
\end{document}