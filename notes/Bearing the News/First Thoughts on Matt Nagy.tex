% Created 2018-01-14 Sun 11:40
\documentclass[11pt]{article}
\usepackage[utf8]{inputenc}
\usepackage[T1]{fontenc}
\usepackage{fixltx2e}
\usepackage{graphicx}
\usepackage{longtable}
\usepackage{float}
\usepackage{wrapfig}
\usepackage{rotating}
\usepackage[normalem]{ulem}
\usepackage{amsmath}
\usepackage{textcomp}
\usepackage{marvosym}
\usepackage{wasysym}
\usepackage{amssymb}
\usepackage{hyperref}
\tolerance=1000
\author{Tom Shannon}
\date{\today}
\title{First Thoughts on Mitch Trubisky}
\hypersetup{
  pdfkeywords={},
  pdfsubject={},
  pdfcreator={Emacs 25.3.1 (Org mode 8.2.10)}}
\begin{document}

\maketitle
\tableofcontents

So first of all, I'm back.  My course at the school where I work is
wrapping up and I've got a bit more time to write than I have had.  I
will probably have to go through the same thing in August and
September next year when I am course director of another new course
but then after that I hope more time will be available to have some
fun.

Since my last post, the Bears have fired \textbf{John Fox} and hired
new head coach \textbf{Matt Nagy}.  Here are some thoughts.

\section{Why Not Someone with a Little Experience?}

I wasn't overly happy about this hire.  I don't have anything against
it but at the same time I'm not too thrilled with it either.

What I see here is manifestation of something that I see a lot
around the league.  When you have a change in leadership in a front
office or in a coaching staff you go out and hire the exact opposite
of what you had before.  I don't know if that that was exactly called
for here.

I don't mind that they hired an offensive head coach (as opposed to
the defensive Fox) and I think it is generally a really good idea to
hire a head coach with a background in coaching quarterbacks.  It's
quarterback-centric league.  But I wouldn't have hired yet another
first time head coach who is super young just because the last guy was
experienced and older.  It has all the markings of an over-reaction to the situation.

John Fox was on his third head coaching job and he'd already gotten
the simple mistakes out of the system.  That's not to say that all of
the decisions that he made were good ones.  It probably didn't help
that he was a conservative head coach and that was, debatably, hurting
the development of rookie quarterback \textbf{Mitch Trubisky} but the
decision that really hurt him the most was his choice of offense
coordinator. \textbf{Dowell Loggains}, as \textbf{Mike Mulligan} at
the \textit{Chicago Tribune}
\href{http://www.chicagotribune.com/sports/football/bears/ct-spt-bears-offseason-changes-mulligan-20171226-story.html}{put
  it}, ran an offense with plays but little in the way of scheme.
That, along with the lack of talent at wide receiver on that side of
the ball, resulted in a fairly low offensive output.

So I'm not saying that experience is everything.  But it certainly can
help avoid a lot of the problems that Matt Nagy is likely to run into
in his first head-coaching job.

So if it were me, I'd have preferred to go with someone with more
experience like \textbf{Josh McDaniels} or \textbf{Pat Shurmur}.
Whether McDaniels wanted to come Chicago and leave a
plum job with the Patriots is debatable.  But my gut is telling me that they
could've persuaded Shurmur to come here if they tried and had been willing to wait a little bit (see below).

\subsection{Are All the Wrong Things All the Right Things?  We'll Find Out.}

As it is the Bears have got a 39 year old head
coach to go with an inexperienced, 40 year old general manager who is frankly doing
many of the things that over the years I have been told are all of the wrong
things for general managers to do.

The first prime example of that is falling in love with Trubisky to
the point where in the 2017 draft he arguably traded away picks to
move from \#3 overall to \#2 overall when he actually didn't have
to.  Pace fell for Trubisky so hard that he felt that he had to do that if there was even the
smallest chance that he might not be able to get him.

Conventional wisdom says that you don't fall in love with draft
prospects.  First rounders are a 50:50 proposition whether you have
fallen in love with them or not. To some extent, you have
to let the chips fall where they may and do the common sense thing.

So it was somewhat disconcerting to see Pace do the exact same thing
again when hiring his head coach.  Pace obviously felt that he had to
rush to hire Matt Nagy in part because the Indianapolis Colts might
have been interested.  Colts general manager \textbf{Chris Ballard}
has a history with Nagy through his Kansas City connections so this
does make some sense but at the same time, conventional wisdom says
that you take your time when you're hunting for a head coach.  You
make your move only after you're absolutely sure that you got the
right guy.  You don't rush it.

My guess is that the Bears also hurried this hire in part they wanted to
get first crack at the assistant coaches that they wanted. Again,
conventional wisdom says not to do that.  Whatever else
you say about John Fox, he assembled a pretty good staff in Chicago.  He was
able to do it, not because he rushed the process, but because he's been
around the league a lot and knows a lot of people.  He has a lot of
connections.

Assembling a staff is probably the most important thing you'll do as a
head coach.  If you hire the right guy, he'll do it right.  On the
other hand, does a young guy like Nagy who has only been a coach since
2008 (and even then it was it as an intern) and who has only
worked for one organization going to have the connections to hire the
right people to to get onto the staff?  We'll find out when the boys
hit the field.

\subsection{The Good:  Copying the Rams Model}

On the generally positive side, there are a lot of similarities in this hiring
to what the Rams did when they hired \textbf{Sean McVay} last year.

Part of the plan in Los Angeles was to get a really good, experienced
defensive coordinator to pair with him.  In the case of the Rams, that was \textbf{Wade
  Phillips}, arguably the best there is right now.

In the Bears case, that guy is \textbf{Vic Fangio} and that's probably
another reason for rushing this hire, perhaps the only legitimate one.
Fangio was under contract with the Bears until Tuesday and getting
Nagy hired quickly gave them a chance to get an offer to Fangio on the
record before he started titling to other teams.  The end result was
that he was hired, presumably as head coach of the defense.

Could they have found a good offensive coordinator if Fangio had
turned them down?  Probably.  And probably they shouldn't have rushed
this hire just to get him.  But having said that, it will be nice for
the players to have the continuity and presumably it keeps much of the
defensive staff intact.  If nothing else, its a proven group.

Finally, and most importantly, they got a good young
quarterback-centric head coach.  A guy who, presumably, will be
Chicago's McVay.  Pace almost certainly had a picture in his mind of
what he was looking for and that picture presumably looked allot like
\textbf{Sean Payton}.  Certainly their backgrounds are similar as each
was a borderline professional quarterback, Payton as a replacement
during the NFL strike in 1982, Nagy as an arena league quarterback.
But the similarities end there as Payton was far more accomplished as
an offensive coordinator when the Saints hired him as their head coach
than Nagy is now.  We shall see if Nagy has Payton's ``fire in the
belly.''

Hiring Nagy could be as good for
Trubisky as hiring McVay was for Jared Goff.  Assuming that Nagy runs
an offense similar to what \textbf{Andy Reid} does in Kansas City, we're
looking at a highly structured West Coast offense where a lot of the
quarterback's decisions will be mapped out.  Its a quarterback
friendly offense where Trubisky will always know what to do and will
have options to get the team into the right play.

Trubisky came to the Bears with a reputation for being very accurate
and we have seen flashes of that on occasion.  The football cognoscente
believe that if he develops consistent mechanics, he'll be a good,
accurate, precision passer that will hit many of the easy, short
passes that the West Coast offense can provide consistently.

Nagy's also got a reputation for being able to adapt to the
characteristics of his quarterback.  He should be able to do better
job than Loggains did of taking advantage of Trubisky's mobility.
We'll probably see a lot more will roll outs and boot legs that will
allow slower developing pass plays to take place and Trubisky to take
off and run if he needs to.

Nagy did a nice job with \textbf{Alex Smith} and, probably more to the
point, \textbf{Pat Mahommes} in Kansas City. We can hope that he
brings that same expertise here and that Trubisky becomes all that the
current regime thinks he can be.

If he does, then the process of hiring Nagy will be characterized as
``decisive'' by future critics.  But for now, it feels like the Bears
are going to have to be a lot luckier than usual to have found the
right guy in such a rushed manner.

% Emacs 25.3.1 (Org mode 8.2.10)
\end{document}
%%% Local Variables:
%%% mode: latex
%%% TeX-master: t
%%% End:
