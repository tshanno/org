% Created 2019-02-17 Sun 09:59
% Intended LaTeX compiler: pdflatex
\documentclass[11pt]{article}
\usepackage[utf8]{inputenc}
\usepackage[T1]{fontenc}
\usepackage{graphicx}
\usepackage{grffile}
\usepackage{longtable}
\usepackage{wrapfig}
\usepackage{rotating}
\usepackage[normalem]{ulem}
\usepackage{amsmath}
\usepackage{textcomp}
\usepackage{amssymb}
\usepackage{capt-of}
\usepackage{hyperref}
\author{Tom Shannon}
\date{\today}
\title{}
\hypersetup{
 pdfauthor={Tom Shannon},
 pdftitle={},
 pdfkeywords={},
 pdfsubject={},
 pdfcreator={Emacs 26.1 (Org mode 9.1.9)}, 
 pdflang={English}}
\begin{document}


\subsection{Bears}
\begin{itemize}
\item \textbf{Brad Biggs} at the \textit{Chicago Tribune} \href{https://www.chicagotribune.com/sports/bears/ct-cb-chicago-bears-mailbag-mitch-trubisky-offensive-line-20191218-33a7ey47aja67p27sh2t5kwube-story.html}{answers your questions}:

\begin{quote}
  ``The Bears will be playing the Chiefs on Sunday, pitting \textbf{Patrick Mahomes} against \textbf{Mitchell Trubisky}. I’m sure that there will be all sorts of comparisons along with what-ifs before and during the game. I wonder, however, what Mahomes would be like right now if he had been drafted by the Bears. Would [head coach \textbf{Matt}] \textbf{Nagy}’s offense have dulled him? Would he have been micromanaged and not the quarterback that he is today for Kansas City? I am curious about the inverse of that thought too. What would Trubisky be like today after spending most of three seasons with the Chiefs under Andy Reid? — Tom H., Chicago

``It’s not a perfect apples-to-apples comparison. But you’re overlooking the fact that Matt Nagy was the offensive coordinator during the 2017 season in Kansas City when Mahomes was being developed as a rookie. Nagy isn’t going to claim to be the guy who made Mahomes the player he is today or the player he was last season when he was the NFL MVP, but he was part of the guy’s development, right? We’re talking about very similar playbooks, too, so the suggestion that Mahomes wouldn’t be a star if he was with the Bears seems off the mark to me. I tend to doubt Trubisky would be significantly better had he landed in Kansas City. Maybe a little bit, but the coach can’t see the field for the quarterback when he’s on the field.''
\end{quote}

Bigg's points are well-taken but its well known that one of the reasons why Chiefs coach \textbf{Andy Reid} handed over play calling duties to Nagy lae that year was so that he cold personally work with Mahommes more.  Its definitely not out of the question that had the two been switched, trubisky wouldn't be considerably more successful as the Chiefs quarterback. 

\item Biggs \href{https://www.chicagotribune.com/sports/bears/ct-cb-chicago-bears-mailbag-mitch-trubisky-offensive-line-20191218-33a7ey47aja67p27sh2t5kwube-story.html}{answers another one}:

  \begin{quote}
  ``Do you think the Bears will look to replace \textbf{James Daniels} and/or \textbf{Bobby Massie}? I know Daniels has played better at guard, but something about the line isn’t working. — @daniel\_larocco

  ``I agree that the offensive line has not played to its potential this season. Massie certainly hasn’t done anything the last three weeks to be downgraded, though. He has been sidelined with a high ankle sprain and hasn’t played. Daniels hasn’t been as good as the Bears would have hoped, but he’s a second-round draft pick and he’s only 22. I don’t see a scenario in which the Bears don’t count on him as a starter for next season. Massie was extended last January, and his 2020 base salary of \$6.9 million is fully guaranteed. He’s not going anywhere either. The question on the line will be \textbf{Rashaad Coward}. He got off to a decent start, but I’m not sure he has really advanced since moving into the starting lineup in place of right guard \textbf{Kyle Long}. The Bears will, at the minimum, need to find some competition that they like beyond just \textbf{Alex Bars}, the undrafted rookie free agent from Notre Dame. They also need to figure out where they want to play Daniels and \textbf{Cody Whitehair} and leave them there. In an ideal world, they’d get a developmental offensive tackle in the draft and maybe even an interior lineman in the middle rounds that they can groom for a year.''
\end{quote}

I would agree with the assessment in that the interior of the line appears to be the problem, specifically both guard positions.  The only thing I would question is whether the Bears should seriously consider replacing Daniels.  Near the end of his second year, Daniels has been part of the problem, not the solution.  Good organizations know when to move on from mistakes.  It might be time for the Bears to do it with him.
\item And \href{https://www.chicagotribune.com/sports/bears/ct-cb-chicago-bears-mailbag-mitch-trubisky-offensive-line-20191218-33a7ey47aja67p27sh2t5kwube-story.html}{another one}:
  \begin{quote}
  ``What do the Bears go next for next season? New QB? New RB? New OL? New TE? New kicker? I think we are good at WR and on defense. — @jojopuppyfish

``They’re going to have to take a hard look at what didn’t go right on offense, which was a lot. I believe they need to consider veteran options at quarterback. I’m not sure the brain trust will agree with that assessment. \textbf{David Montgomery} has done better as the season has gone along and I believe he will be back as the starter next season. They will probably look for one starter on the offensive line, but as I have detailed in previous questions, I don’t expect wholesale changes to the personnel there. The Bears have to get a lot more from the tight end position, and they’re going to need to cover themselves in case \textbf{Trey Burton}, who is guaranteed \$4 million of his \$6.7 million base salary, cannot produce. I believe they at least need to have \textbf{Eddy Pineiro} compete for the kicking job next season. I believe the Bears will have to address the wide receiver position as well. \textbf{Anthony Miller} has really stepped up in the second half of the season and he looks like he’s going to be a pretty good player. But other than \textbf{Allen Robinson}, they are thin at that position, and I think they need to move on from \textbf{Taylor Gabriel} and come up with a quality option. They’re using \textbf{Cordarrelle Patterson} a lot, but he’s little more than a decoy on offense. They need another wide receiver who can produce on the outside and, at minimum, push \textbf{Riley Ridley} for playing time. The defense should be good. They’ve got some moves to make there, but the unit remains strong. It will be a busy offseason and an interesting one.''
\end{quote}

Biggs literally read my mind and I agree with every word of this.  It's also very possible that they'll be seeking help at inside linebacker with both \textbf{Danny Trevathan} and \textbf{Nick Kwitkowski} entering the unrestricted free agent market.  However, with limited draft picks and cap space, there's only so much the Bears are going to do.

I'm reminded of one of the more profound statements that former Bears defensive coordinator \textbf{Vic Fangio} made about his unit a couple years ago.  The Bears need more from their ``so called good players''.  At some point you are going to have to coach up and improve the talent you have at any or all of these positions if you want to get better.  It's only a question of which are the positions that are the most in need of new personnel.

I'm going to guess that the Bears roll with Trubisky at quarterback and pray that it all finally falls together for him, as unlikely as that seems to be to some on the outside looking in.  I suspect that they won't bring in serious competition at guard with Bars waiting in the wings.  A lot will depend on who they like in the draft, though.  We shall see.
 
\item And \href{https://www.chicagotribune.com/sports/bears/ct-cb-chicago-bears-mailbag-mitch-trubisky-offensive-line-20191218-33a7ey47aja67p27sh2t5kwube-story.html}{another one}:

  \begin{quote}
  ``Based on the fact that the offense has never really appeared to be in sync, do you think Matt Nagy will change his approach toward the amount of live game actions some of the key players (Mitch Trubisky, for example) will see during the preseason? — @mike32198768

``Everything has to be on the table for consideration when Nagy maps out a plan for 2020, including what the goals are for the preseason games. I am sure Nagy will assess his approach in preseason. Will that mean we see starters in preseason? I don’t know. I do know the goal is to get to games that are meaningful with a healthy roster, something the Bears have accomplished the last two years. I also firmly believe that you cannot pin the struggles of the offense this season on missed action in preseason. If you evaluate how much some starters play for other teams, the Bears’ front-line guys missed out on maybe 60 snaps in preseason, 75 tops. You can’t tell me that those snaps would have made a difference for this team in Week 4, Week 8, Week 12 or now.''
\end{quote}

No.  But I absolutely believe they would have made a difference in weeks 1 and 2.  And a better performance against the Packers would have made a huge difference in how this year went.

Nagy failed to properly evaluate Trubisky coming into this season and it seriously damaged their chances this year.  He chose to pile more on him and ``move from offense 101 to 202'' as he put it instead of recognizing that Trubisky needed to solidify his gains from last season first.  It was one of many decisions that Nagy made this season that didn't work out after having virtually everything go right last season.  It was probably the one that had the biggest impact as Trubsky regressed and struggled to recover for half the year.

Forty snaps in the preseason might well have shown Nagy that he was making a mistake.  It certainly didn't take 40 snaps in the regular season to show the rest of us.

There's a reason why teams have always played their starters in the preseason to at least some extent.  The Bears in general, and Trubisky in particular, were woefully unprepared to play at game speed when the season started.  Even given that conditions aren't exactly like real games, playing more in the preseason against starting caliber players on the other team probably would have helped.

Whether you agree with that or not, something has to change in the way the Bears prepare for the season.

\end{itemize}

\subsection{One Final Thought}

Biggs and \textbf{Colleen Kane} \href{https://www.chicagotribune.com/sports/bears/ct-cb-chicago-bears-allen-robinson-cordarrelle-patterson-20191219-aixloh5vi5auxfbrxpxfioulru-story.html}{talk about the uphill battle Bears punt returner \textbf{Tarik Cohen} faces} against a very good Chiefs special teams unit:

\begin{quote}
``Cohen is fourth in the NFL, averaging 9.2 yards per return, but faces a tough task against the Chiefs. They are allowing only 4.2 yards per return, and the long return against \textbf{Dave Toub}’s unit this season is 11 yards.''
\end{quote}

Toub is flat out the best special teams coach in the league.  It really is a shame he hasn't gotten a chance at a head coaching job.  Good special teams coaches have to be both talented and resourceful because they are always dealing with back ups at the bottom of the roster.  I'm convinced Toub would be a good one.  Perhaps one day someone will give him the chance that former special teams coach \textbf{John Harbaugh} got to show what he could do with the perennial contender Baltimore Ravens.
\end{document}
