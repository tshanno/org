\begin{itemize}
\item  \textbf{Mark Potash} at the \textbf{Chicago Sun-Times}  \href{https://chicago.suntimes.com/sports/mark-potash-analyzing-the-bears-2019-draft-class/}{reviews the Bears 2019 NFL draft}:

\begin{quote}
In his fifth draft, general manager \textbf{Ryan Pace}:

Didn’t change. He was aggressive as usual, but this time he’s playing from strength instead of weakness — fortifying a playoff team instead of building from the ground up. Even if you consider \textbf{Khalil Mack} and \textbf{Anthony Miller} part of this draft, Pace can’t afford to strike out after that, yet he took some chances on players with big upsides. It’s risky. But with this roster, Pace isn’t playing with scared money.
\end{quote}

As I do annually, I did a mock draft with reps from the other NFL teams for a \href{https://t.co/VWwtXBGEga}{podcast}.  I was on last, as the Bears didn't have a pick until the third round.  I thought it was ironic that I was on with the Saints rep who, as usual, didn't have a pick in the first round either.

When Ryan Pace came from New Orleans one of the firs things I noticed was that the Saints were always extremely aggressive with the resources that they had available.  Pace has definitely been applying whatever he learned there.

Like most Saints fans, I think we're going to have to get used to being chronically short on draft picks and cap space.  If it means being a consistent contender, I think I can live with it.

\item The writers are the \textit{Chicago Tribune} \href{https://www.chicagotribune.com/sports/football/bears/ct-spt-cb-chicago-bears-nfl-draft-picks-2019-story.html}{describe the new Bears running back \textbf{David Montgomery}'s positive traits}:

\begin{quote}
  ``Montgomery is a well-rounded back who will contribute immediately and might even start the season opener.''
  
``Where to begin? On the field, Montgomery has great instincts, vision, balance and lateral agility. He’s a human pinball. He led the nation in forced missed tackles in each of the last two seasons, according to \textit{Pro Football Focus}.''  
\end{quote}

One thing you never hear about in these descriptions is the back's ability to block.  That's because they are rarely asked to do it in college.  But that will be the most important factor which determines whether Montgomery will start right out of the gate.

No one is going to put any running back out there if they think he's going to get the quarterback killed.

\item \textbf{Phil Rosenthal} and \textbf{Tim Bannon} at the \textit{Chicago Tribune} \href{https://www.chicagotribune.com/sports/football/bears/ct-spt-bears-nfl-draft-winners-losers-espn-20190429-story.html}{cover the draft's winners and losers and don't while not being stupid about it}.  Dolphins fans will want to skip to the last few.

  In the mean time I'd like to add my own winner:  \href{https://en.wikipedia.org/wiki/Joel_Klatt}{\textbf{Joe Klatt}} at the NFL Network.  Klatt usually covers college football for FOX but he came on and did a great job during Day 2 of the draft breaking down the picks.  I spent most of Day 3 wishing he was still there instead of \href{https://en.wikipedia.org/wiki/Peter_Schrager}{\textbf{Peter Schrager}}, who seemed to be there more for the entertainment value.
  
\item \textbf{Scott Bordow} at the \textit{Arizona Republic} \href{https://theathletic.com/945667/2019/04/25/2019-nfl-draft-live-tracker-dane-brugler/}{reviews the pick of quarterback Kyler Murray} for \textit{The Athletic}:

\begin{quote}
  Cardinals general manager \textbf{Steve Keim} said he didn’t want to take \textbf{Kyler Murray} after trading up to get Josh Rosen in the first round last year but he was won over by Murray’s talent. This is a gamble for Arizona. Few teams had Murray as the No. 1 player in the draft and it’s fair to wonder if the Cardinals would even have considered Murray if they hired anyone other than \textbf{Kliff Kingsbury} as coach. This will either work out spectacularly for Arizona or cost Keim and Kingsbury their jobs some day.
\end{quote}

My guess is that Bordow is right.  The Cardinals don't take Murray if anyone other than Kingsbury is the head coach.  The reason is simple.  Murray fits what the Cardinals want to do out of the box whereas any other coach outside of Seattle would have had to change their entire offense to make Murray work.  Murray's size makes him less than suitable for an offense that relies on sitting in the pocket and finding the open receiver.

I tried to figure out where the next likely landing place for Murray was if he got by the Cardinals.  My guess is that it would have been a long way down the list.  Murray not only landed in the perfect spot to take advantage of his talents.  He may have landed in the only spot.  
\end{itemize}
