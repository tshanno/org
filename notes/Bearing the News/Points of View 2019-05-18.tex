% Created 2019-02-17 Sun 09:59
% Intended LaTeX compiler: pdflatex
\documentclass[11pt]{article}
\usepackage[utf8]{inputenc}
\usepackage[T1]{fontenc}
\usepackage{graphicx}
\usepackage{grffile}
\usepackage{longtable}
\usepackage{wrapfig}
\usepackage{rotating}
\usepackage[normalem]{ulem}
\usepackage{amsmath}
\usepackage{textcomp}
\usepackage{amssymb}
\usepackage{capt-of}
\usepackage{hyperref}
\author{Tom Shannon}
\date{\today}
\title{}
\hypersetup{
 pdfauthor={Tom Shannon},
 pdftitle={},
 pdfkeywords={},
 pdfsubject={},
 pdfcreator={Emacs 26.1 (Org mode 9.1.9)}, 
 pdflang={English}}
\begin{document}

\subsection{Bears}
\begin{itemize}
\item  \textbf{Dan Wiederer} at the \textit{Chicago Tribune} \href{https://www.chicagotribune.com/sports/bears/ct-spt-bears-kerrith-whyte-lane-kiffin-20190514-story.html}{asks Florida Atlantic head coach \textbf{Lane Kiffin} about Bears sixth round pick \textbf{Kerrith Whyte}}:

\begin{quote}
Wiederer: Do you see any limitations in his game or areas he’s going to have to either really work on or work around to be reliable at the NFL level?

Kiffin:  Pass protection. With the bigger players on that level, that’ll be more of a challenge. And being the backup here, it’s not like he got a million reps on film of that. So that’s going to be something he’ll have to work at if they’re going to want to use him on third down especially.
\end{quote}

No surprise there.  No team will put a running back out on the field who is going to get the quarterback killed.  Kiffin, a former head coach of the Oakland Raiders, would know that as well as anyone.  And most good college running backs haven't been asked to do it much. 

The good news is that Kiffin thinks Whyte is a really good pass receiver.  But the bet here is that if Whyte sees the field much for the Bears his rookie year, it will be as a kick returner, a position that he excels at.

\item \textbf{Rich Campbell} at the \textit{Chicago Tribune} \href{https://www.chicagotribune.com/sports/bears/ct-spt-bears-stephen-denmark-valdosta-state-20190513-story.html}{interviews \textbf{David Rowe}}, Valdosta State's defensive backs coach, about new Bears cornerback \textbf{Stephen Denmark}:

\begin{quote}
Rowe:  ``To bring you back to that last game (his junior year), we had a staff meeting and we were trying to talk about guys who could come over and help us because our numbers were down.

``The receivers coach and the head coach were like: Stephen Denmark needs to be over there. They were trying to sell us on: You should see the way he gets out of breaks. He’s 6-2 and can get out of breaks as well as all our small guys. So his feet are really good.

``Just naturally, him coming over to a new position, he’d have to get better at backpedaling and all that, which we didn’t do a lot of that. He was able to do it; it just took some time and work. He busted his (tail) and got it done.''

Campbell:  ``So did he play with a lot of vision, backed off a bit?''

Rowe:  ``No, we played majority press coverage. Turn and run with a guy. There were some situations where he did play a bail third, and he did a pretty good job with that stuff.''
\end{quote}

I'm really wondering why the offensive coaches pushed for Denmark to make this switch.  Big, athletic wide receivers presumably don't' grow on trees, especially at Valdosta State.  My conclusion is that it my have had to do with his hands and his ability to catch the ball.

The picture that Campbell and Rowe paint is of a very raw prospect who hasn't done much other than press coverage, a technique that requires athleticism with very little knowledge of the defense or of the awareness that is necessary to play other techniques.

Bottom line, Denmark only played a year at cornerback at a low level college.  Bears fans probably shouldn't expect an immediate contribution as he probably has a long way to go.  Making the roster would likely be an accomplishment for him.

\item \textbf{Patrick Finley} at the \textit{Chicago Sun-Times} \href{https://chicago.suntimes.com/sports/bears-jesper-horsted-sign-undrafted-free-agent/}{reports that the Bears have signed tight end \textbf{Jesper Horsted}}.

At 6-4, 225 pounds Horsted was listed as a wide receiver at Princeton.  He will likely compete to back up \textbf{Trey Burton} at the U-tight end rather than at the in-line blocking Y-tight end position.

The latter is a weak spot in the Bears offense as \textbf{Adam Shaheen} tries to break out in his third year with the Bears.  Shaheen hasn't developed in part because his career to this point has been marred by injuries.

Right now Shaheen's primary competition comes in the form of veteran back up \textbf{Ben Braunecker} and undrafted free agents \textbf{Ian Bunting} and \textbf{Dax Raymond}.

One can only conclude that the Bears are a lot more comfortable about Shaheen being given this job than many of the fans are.

\item \textbf{Darin Gantt} at \emph{profootballtalk.com} \href{https://profootballtalk.nbcsports.com/2019/05/17/bears-had-tryout-punter-kicking-field-goals-during-minicamp/}{comments on the Bears ``clown car'' approach to finding a new kicker}.  They did everything they could to put the 9 kickers (including the tryout punter) in pressure situations in thier recent minicamp and drew a lot of attention to the situation by doing so.

\begin{quote}
``It’s easy to view [the approach] as obsessive, if not excessive. While it’s clearly important for a team to find a reliable kicker, the way the Bears are setting up this search also puts the job in more of a spotlight than it naturally brings. Now, as soon as the next poor soul to hold that job misses a kick and the Bears lose a game, he’ll become the focus of the larger failure. They made Parkey a pariah, partly for the miss and partly for his television appearance in the aftermath, and now they’re ensuring his replacement will be under even more pressure.''
\end{quote}

You won't find many people that are too sympathetic in this situation.  If you have a kicker line up for a game winner in a Super Bowl, you would be hard pressed to find a situation more pressure packed.  You'd like to have a guy in that role that you are confident in.  I think most people will have a hard time criticizing the Bears for doing everythig they can to make sure that they have a guy who will react the right way.
\end{itemize}

\subsection{Elsewhere}

\item \textbf{Mike Florio} at \emph{profootballtalk.com} \href{https://profootballtalk.nbcsports.com/2019/05/17/rumors-fly-of-the-jets-pursuing-peyton-manning/}{explains what the Jets might have in mind for their general manager position}:

\begin{quote}
``Jets CEO and chairman \textbf{Christopher Johnson} wants a 'great strategic thinker' to run the football operation. He needs someone who can work with coach \textbf{Adam Gase}. And at the intersection possibly resides one and only one name.

\textbf{Peyton Manning}.
\end{quote}

This sounds to me more like the media connecting dots than a realistic possibility.  But I've been surprised before.

I love Peyton Manning but a general manager?  I'm not a big fan of having people without a background in personnel in that role, let alone someone with no front office background at all. It almost never works out.  The latest example is in San Francisco where rumor has it that \textbf{John Lynch}, who also had no front office experience, and head coach \textbf{Kyle Shanahan} are rumored to be on the outs.

I don’t like the direction the Jets are taking.

\item Florio also \href{https://profootballtalk.nbcsports.com/2019/05/17/prosecution-appeals-suppression-order-in-kraft-case/}{explains why prosecutors are appealing a ruling in the case against \textbf{Robert Kraft}}, who is accused of solicitation in Florida.

\begin{quote}
``Multiple judges have ruled that the “sneak and peek” video surveillance violated the law by undertaking no effort to minimize the intrusion on the privacy of innocent persons who were simply getting massages. If the appellate courts don’t overturn these rulings, there will be little or no evidence against Kraft — unless prosecutors can persuade the alleged providers of prostitution to “flip” on their alleged customers.''
\end{quote}

I have no interest in this case except that it bothers me when someone tries to legally get off the hook based upon technicalities.  I know he did it.  You know he did it.  The lawyers know he did it.

Kraft is the owner of a franchise where players are constantly told to be accountable for their actions.  Is this accountability?  It might be the reality of the world we live it.  But I call it hypocrisy.

In any case, the situation \href{https://profootballtalk.nbcsports.com/2019/05/16/if-robert-kraft-is-exonerated-what-happens-next/}{puts the league in a bind}.  They  haven't hesitated to suspend players who are obviously guilty but who have not been legally convicted, often because they paid off the victim.  Pittsburgh quarterback \textbf{Ben Roethlisberger}'s \href{https://en.wikipedia.org/wiki/Ben_Roethlisberger#Sexual_assault_allegations}{2010 suspension after sexual assault allegations} is a good example.

I think its fair to say that although the league has some morally upright fans who strongly disapprove, solicitation isn't really considered to be a big deal to most in modern American society.  It certainly doesn't rise to the level of sexual assault or similar offenses.  But in terms of obvious guilt or innocence beyond the legal ramifications, there are players who are going to be watching this situation closely to see if Kraft is held to the same standard.


\subsection{One Final Thought}
\begin{figure}
  \centering
  \includegraphics[width=\textwidth]{/Users/tshanno/Library/Mobile Documents/com~apple~Preview/Documents/ReggieBushTweet.png}
\end{figure}
\end{document}