% Created 2019-02-17 Sun 09:59
% Intended LaTeX compiler: pdflatex
\documentclass[11pt]{article}
\usepackage[utf8]{inputenc}
\usepackage[T1]{fontenc}
\usepackage{graphicx}
\usepackage{grffile}
\usepackage{longtable}
\usepackage{wrapfig}
\usepackage{rotating}
\usepackage[normalem]{ulem}
\usepackage{amsmath}
\usepackage{textcomp}
\usepackage{amssymb}
\usepackage{capt-of}
\usepackage{hyperref}
\author{Tom Shannon}
\date{\today}
\title{}
\hypersetup{
 pdfauthor={Tom Shannon},
 pdftitle={},
 pdfkeywords={},
 pdfsubject={},
 pdfcreator={Emacs 26.1 (Org mode 9.1.9)}, 
 pdflang={English}}
\begin{document}

\begin{itemize}
\item \textbf{Patrick Finley} at teh \textit{Chicago Sun-Times} \href{https://chicago.suntimes.com/sports/nfl-draft-2019-bears-ryan-pace-five-things-we-learned/}{describes his take aways} from GM  \textbf{Ryan Pace}'s pre-draft press conference.

  \begin{quote}
    ``Last week, NFL Network reported that the Raiders sent their scouts home to protect the secrecy of their draft room. That’s not the case at Halas Hall, where Pace said all the team’s scouts — plus [head coach \textbf{Matt}] \textbf{Nagy} — will occupy their new state-of-the-art draft center.

``'I think, by nature, everybody in my role is a little paranoid,' Pace said. 'We’ve been together for a long time now, and we have a tight group. Continuity. We’re all in there together. And they’re going to come in tonight and we talk through scenarios. So it’s still very collaborative all the way through.'''
\end{quote}

One of the things I noticed this year was how locked down the Bears were in terms of what they were doing in prepratation for this draft.  There was a time when a fan could track most of the pre-draft player visits to Halas Hall.  That wasn't the case this year.

Pace has generally done a good job of keeping things under wraps as general manager.  He seems to work harder than most to prevent leaks and I think he gets particularly angry when they happen despite that, as they did when the Bears were negotiating with \textbf{Roquan Smith} last year and his agent leaked information to \textit{profootballtalk.com} about stipulations he was seeking in the contract.

\item \textbf{Adam Jahns} at \textit{The Athletic} \href{https://theathletic.com/941781/2019/04/23/making-sense-of-what-ryan-pace-said-in-his-annual-pre-draft-press-conference/?redirected=1}{also interprets what Pace said at the press conference}.  Jahns addresses the possibility that the Bears might take a running back:

  \begin{quote}
    ``It’s widely viewed as a good year for mid-round running backs. The Bears, including coach Matt Nagy, have spent considerable time looking at them, too.

``In \textbf{Dane Brugler}’s recent mock draft  for \textit{The Athletic}, he projects that five running backs will go in the third round.

``Moves in free agency also can shroud a team’s intentions in the draft. Don’t forget that the Bears signed \textbf{Mike Glennon} before drafting [quarterback \textbf{Mitch}] \textbf{Trubisky}.

``The surprise of the Bears’ draft might be them passing on a back in the third round. Still, it’s safe to assume that one will be added during the draft. Pace drafted [\textbf{Jordan}] \textbf{Howard} in the fifth round in 2016, while \textbf{Jeremy Langford} was taken in the fourth in 2015.
\end{quote}

When looking over the Bears draft needs, running back certainly ranks high on the list.  But there are a few other positions that it would be no surprise in the Bears addressed.  \textbf{Ha Ha Clinton-Dix} is on a one year prove it deal and with a contract extension for \textbf{Eddie Jackson} on the horizon a developmental prospect that could be ready to start next year should be at least on the list of possibilities.  Same with cornerback where the Bears are paying \textbf{Kyle Fuller} with \textbf{Prince Amukamara} on the other side.  Amukamara \href{http://bearingthenews.com/blog/2019/02/18/bears-should-be-uneasy-about-amukamaras-late-season-performance-and-other-points-of-view/}{had 7 pass interference or defensive holding calls against him last year} and all came in the second half of the season including one playoff game. Three came in the last two games against the Vikings and the Eagles.

However, if I were to pick one position to keep an eye on, it would continue to be tight end.  \href{http://bearingthenews.com/blog/2019/03/15/the-bears-were-uncommonly-healthy-and-will-have-to-overcome-adversity-to-succeed-in-2019-and-other-points-of-view/}{I continue to assert} that the tight end class is deep in the middle rounds and there's more than the usual degree of possibility that the Bear could pick up a big, versatile tight end that could push \textbf{Adam Shaheen} for the starting role.
  
\item \textbf{Brad Biggs} at the \textit{Chicago Tribune} \href{https://www.chicagotribune.com/sports/football/bears/ct-spt-bears-robbie-gould-49ers-trade-biggs-20190423-story.html}{does a good job of breaking down the kicker situation},  after news broke Tuesday that former Bears kicker \textbf{Robbie Gould} has requested a trade.  He has reportedly informed the 49ers that he will no longer negotiate a multiyear contract and that he is no lock to show up for the start of the regular season after the 49ers placed the franchise tag on him.

  \begin{quote}
    ``It’s unknown whether the Bears, with three unproven options on the roster, would have interest in acquiring Gould, even though he has been nearly automatic since they surprisingly released him during final cuts in 2016. It’s clear Gould would welcome the opportunity to play for the Bears again. He lives in the area and brought his family to the playoff loss to the Eagles that ended when \textbf{Cody Parkey}’s 43-yard field-goal attempt was partially tipped at the line of scrimmage and then hit the left upright and crossbar.

    ``What can’t be overlooked is that performance could not have been the driving factor in the Bears’ original decision to cut Gould and replace him with \textbf{Connor Barth}.  Gould didn’t have his best season in 2015, general manager Ryan Pace’s first with the Bears, but he made 33 of 39 field goals (84.6 percent), which at the time was right in line with his career average.''

    ``Pace has aggressively addressed nearly every area to restore the Bears into contenders. Bringing Gould back would provide a daily reminder of how Pace and his staff have gotten the kicker moves wrong at just about every turn.

``The answer to whether Gould can come home again probably lies in the explanation of why he had to go in the first place.''
\end{quote}

I doubt very much that the 49ers are all that worried about this.  I don’t think kickers need to be at training camp and I can’t see him skipping the season.

As Biggs imples, there's a reason why Gould was let go by the Bears in 2015 and it wasn't performance.  It seems clear that he's a high maintenance player.  Its possible that with another coaching staff in place that Pace would consider bringing him back but I don't think the Bears are going to want to commit the cap space to him with an extension for Whitehair pending.

This is much ado about nothing at a time when the Bears first pick is in the third round and there's little in the way of news to report.


\item \textbf{Darin Gantt} at \textit{profootballtalk.com} passes on the fact that the departure of \textbf{Jake Ryan} to Jacksonville last month closes the book on the entire Packers 2015 draft class.

  \begin{quote}
    ``First-round defensive back \textbf{Damarious Randall} was traded to Cleveland a year ago, for backup quarterback \textbf{DeShone Kizer}. Second-round cornerback \textbf{Quinten Rollins} never made much of an impact, and they tried to convert him to safety a year ago before releasing him. He made a brief appearance with the Cardinals last fall but isn’t on a roster now.

Third-round wide receiver \textbf{Ty Montgomery} was converted to running back, and released last fall after fumbling a kickoff he was supposed to down in the end zone.

After Ryan in the fourth round came quarterback \textbf{Brett Hundley} (who signed with Arizona having been previously pawned off on the Seahawks), fullback \textbf{Aaron Ripowski} (who was signed to a future contract by the Chiefs this offseason), defensive tackle \textbf{Christian Ringo} (who has bounced around to a second stint with the Bengals), and tight end \textbf{Kennard Backman} (who appeared in seven games as a rookie).
  \end{quote}

  Much has been made of the decline in talent in the latter years of \textbf{Ted Thompson}'s reign as Green Bay GM.  But I'm going to cut Thompson a little break.  A very little break.

  Talented or not, I saw very little difference between the draft classes that Thompson produced late in his career compared to early in his career.  I thought the blame for the decline of the franchis lies elsewhere.

  First, they aren't developing talent like they used to.  Second, quarterback \textbf{Aaron Rogers} flat out refuses to get rid of the ball on time, prefering to run around and try to make plays raather than throwing with anticipation into tight windows as he did during his best days.

  It's going to be very interesting to see if that franchise does better under new GM \textbf{Brian Gutekunst} or if they remain stuck in the mud with many of the same coaches and the same quarterback.

  \subsection{One Final Thought}

\item Biggs also \href{https://www.chicagotribune.com/sports/football/bears/ct-spt-bears-mailbag-jordan-howard-trade-20190403-story.html}{answers your questons}:

  \begin{quote}
    ``We often saw Mitch Trubisky overthrow wide open receivers. What can you tell us about what he’s doing to improve his accuracy? — @dg122985

``Accuracy is something football folks will tell you a quarterback either has or doesn’t have. Guys who aren’t accurate from the start usually don’t become super accurate. The good news is Trubisky set the Bears’ single-season record for completion percentage last year at 66.6. Yes, he missed some open guys downfield. Trubisky has worked with wide receivers this offseason, which could help when they get back to Halas Hall. It was a new offense for everyone last year, and with the players not having to go through that process again, you would expect things to be a little more crisp. That could make him more accurate.''
\end{quote}

I think \href{https://www.chicagotribune.com/sports/football/bears/ct-spt-bears-matt-nagy-ryan-pace-nfl-owners-meetings-20190326-story.html}{Nagy's thoughts on Trubisky's occasional poor throws} at the NFL owners meetings are particularly relevant:

\begin{quote}
  ``I feel strongly about this. None of it was a physical thing. It was all just him learning where to go with the ball. See, your clock goes so fast that everything you see … Boom! Ball’s out. You know? When you know what’s going on and things become slower, you can make that more accurate throw. None of it was physical. It was all just mentally learning the offense. And the other guys too. Sometimes the quarterback makes a throw and it looks like it’s a bad throw or a poor throw, but it was a terrible route. Right? You guys don’t know that though. And I’m not going to call a guy out in the media.''
\end{quote}

I think this is spot on.  It was very evident that Trubisky got more accuarte last year as he got more comfortable in the offense.  People don't think about it but your mind and body are connected and when one isn't right, the other isn't going to be right either.

I look for Trubisky to continue to progress as the game slows down for him more and more and he gets more and more mentally in tune with what is going on out on the field.  If his body follows, we could finally start to see the North Carolina quarterback who was so accurate that it was an event whenever the ball hit the ground in practice.

\end{itemize}
\end{document}