% Created 2019-02-17 Sun 09:59
% Intended LaTeX compiler: pdflatex
\documentclass[11pt]{article}
\usepackage[utf8]{inputenc}
\usepackage[T1]{fontenc}
\usepackage{graphicx}
\usepackage{grffile}
\usepackage{longtable}
\usepackage{wrapfig}
\usepackage{rotating}
\usepackage[normalem]{ulem}
\usepackage{amsmath}
\usepackage{textcomp}
\usepackage{amssymb}
\usepackage{capt-of}
\usepackage{hyperref}
\author{Tom Shannon}
\date{\today}
\title{}
\hypersetup{
 pdfauthor={Tom Shannon},
 pdftitle={},
 pdfkeywords={},
 pdfsubject={},
 pdfcreator={Emacs 26.1 (Org mode 9.1.9)}, 
 pdflang={English}}
\begin{document}

\begin{itemize}
\item \textbf{Brad Biggs} at the \textit{Chicago Tribune} \href{https://www.chicagotribune.com/sports/football/bears/ct-spt-bears-mailbag-stephen-denmark-david-montgomery-biggs-20190501-story.html}{answers your questions}:

\begin{quote}
  ``Does \textbf{Adam Shaheen} have a future with the Bears? Plagued by injuries first two seasons, I’m wondering if they’ll go in different direction, especially with capable free agents like the guy from Utah State. — @chuckietwoglove

``I can tell you that when \textbf{Matt Nagy} was asked about Shaheen at the NFL owners meeting, he was very positive. There is no question durability has been an issue with Shaheen, and if he can stay on the field, the Bears believe he can really help the offense. Let’s tap the brakes on the idea he will be replaced from the get-go by \textbf{Dax Raymond}, the undrafted rookie from Utah State. The Bears like Raymond and believe he has a chance to stick, otherwise they would not have guaranteed him $45,000 ($15,000 signing bonus with \$30,000 base-salary guarantee). But 32 teams passed on Raymond in the draft. The Bears still have an investment of a draft pick, money and maybe most importantly time in Shaheen. They’re not cutting the cord on him now.''
\end{quote}

One of the bigger surprises of the Bears draft has to be that they didn't draft a tight end from what looked like a deep class in the middle rounds.  The Bears have evidently decided to roll with Shaheen.  Raymond will evidently compete to back him up with an unsigned free agent.

Despite Nagy's comments, that second, inline tight end position has to be an area of mild concern (see below).  Setting aside the fact that he can't stay healthy, Shaheen hasn't shown anything but potential, yet.

\item \textbf{Rich Campbell} at the \textit{Chicago Tribune} \href{https://www.chicagotribune.com/sports/football/bears/ct-spt-bears-david-montgomery-position-coach-20190430-story.html}{interviews Iowa State coach \textbf{Nate Scheelhaase}} on the characteristics of new Bears running back \textbf{David Montgomery}:

\begin{quote}
``He’s a really good inside-zone runner. He has a good feel, especially in the shotgun, of just how the zone moves and how things feel, which is probably why Coach [\textbf{Matt}] \textbf{Nagy} and Coach [\textbf{Mark}] \textbf{Helfrich} and those guys were really interested.

``It was funny. There were a lot of teams interested in David, but the teams that seemed to be the most interested all came from that same tree. It was the Chiefs, Bears, Eagles, the Colts. Those coaches that came from that same tree, they have a bunch of gun runs that they run really well.''
\end{quote}

It's worth noting that the Bears were likely trading up to get ahead of the Bills, who were evidently looking for a running back and selected Florida Atlantic's \textbf{Devin Singletary} with the pick after Chicago's.  So there was likely something of a consensus amongst a lot of teams that he was the next best running back on the board.  Certainly the Bears suspected that the Bills would have taken him.

This article is worth reading.  There was a lot of good information beyond the usual hype.

\item Campbell quotes Bears head coach \textbf{Matt Nagy} as \href{https://www.chicagotribune.com/sports/football/bears/ct-spt-bears-draft-observations-david-montgomery-20190429-story.html}{he talks up Montgomery}:

\begin{quote}
  ``'Every running back has his own strength as far as what route he can run,' Nagy said Saturday, 'whether … he’s a bigger target or he’s super fast. Some running backs you can put them (as a wide receiver with an empty backfield) and they are better on shallow crosses or better with read routes, and it’s not a vertical route or a post route where they beat you with speed. They just have a good knack at that.

``'It’s not always in an empty set. You (deploy) a guy out of the backfield, now (there are) mismatches, right? Is he against a safety or is it a linebacker? Are they playing base defense versus a guy we think we can get an advantage in the pass game? Great. Are they going to play nickel or dime and dare us to run the ball? Well, when you have guys that can play all three downs, it’s nice for the play-caller, and it’s nice for the offense.'''
\end{quote}

It's all great in theory.  But there's a legitimate question as to what kind of mismatch Montgomery will actually be out of the back field.  With a lack of straight line speed, one could argue that, even lined up in a base defense, linebackers might have little trouble keeping up with Montgomery or, for that matter, \textbf{Mike Davis}, the presumed RB2.

Certainly defensive backs will have little trouble keeping up should defenses feel that they can get away with playing nickel against the run (as they frequently do nowadays).  In that respect, the lack of a blocking tight end who can also go out for the occasional pass route could also burn the Bears badly if Adam Shaheen doesn't develop this off season.  Campbell comments further on teh situation:

\begin{quote}
  ``[B]ecause they didn’t draft a tight end, let’s recognize the urgency for Shaheen to play up to his draft pedigree.

``Remember, Pace drafted him 45th overall in 2017. That’s a huge investment, especially in a Division II player. For as much credit as the Bears deserve for developing their two fourth-rounders in that draft — \textbf{Eddie Jackson} and [\textbf{Tarik}] \textbf{Cohen} — Shaheen has lagged. He always was going to require seasoning for his blocking technique and route running, and missing 10 games last season was unfortunate. Now he’s a noteworthy wild card in the offense’s outlook this season.
\end{quote}

Arguably Shaheen's play will be at least as big of a factor determining whether defenses feel forced to play in one formation or another depending upon his ability to both block and run pass routes.  Right now I would say he's not a factor in that decision at all.

In any case, unless Montgomery's shiftiness and ability to break tackles actually results in big plays despite his lack of speed, the Bears could be looking at a situation where they have a couple of work horse backs who can get them yardage on the ground but where their only true mismatch out of the back field will once again be Cohen.

\item If I read \href{https://www.chicagotribune.com/sports/football/bears/ct-spt-bears-david-montgomery-position-coach-20190430-story.html}{one more article about kickers} I'm going to be sick.  Man...
  
\item \textbf{Dave Hyde} at the \textit{South Florida Sun-Sentinal} \href{https://www.sun-sentinel.com/sports/dave-hyde/fl-sp-hyde10-dolphins-draft-thoughts-20190429-story.html}{notes how well the Dolphins are apparently set up for the 2020 NFL draft}:

\begin{quote}
The Dolphins pumped a lot of resources into the 2020 draft this offseason and now have 12 draft picks.

1 — 1st round

2 — 2nd round (theirs and New Orleans)

2 — 3rd round (theirs and \textbf{Ja’Wuan James} compensatory pick)

2 — 4th round (theirs and Tennessee’s from \textbf{Ryan Tannehill} trade)

2 — 5th round (theirs and \textbf{Cam Wake} compensatory pick)

2 — 6th round (theirs and \textbf{Robert Quinn} trade)

2 — 7th round (Kansas City from \textbf{Jordan Lucas} trade)
\end{quote}

I noted the optimism with which Hyde anticipated getting the compensatory picks for James and Wake.  In particular, I'm not convinced that losing James is going to bring a third round compensatory pick but let's assume that he will.

The list highlights one of the changes I think we can anticipate in the new NFL labor contract as the old deal expires in 2 years.

The compensatory pick system was set up to help ameliorate the damage done when a player leaves via free agency.  they weren't meant to completely compensate for the loss, only to make it less disastrous for a team when they lose a particularly valuable asset.  But the system has developed far beyond that now.

The compensatory picks that teams get for getting a free agent go are so valuable that teams are often motivated to game the system by purposely letting the player go and taking the pick instead.  Consider the case of James.  Does anyone think there's any chance that James would have brought a third round pick in a trade before he became a free agent?  Wold anyone have given a fifth round pick for a 37 year old Cam Wake?

The pick system is currently being used for a purpose that it was not intended for -i.e. actually rewarding a team for not signing a player in free agency.  That's something the NFLPA cannot let stand and, given the justice of their case, I have to believe that the league won't fight too hard against changing the system.

\item Colts owner \textbf{Jim Irsay} \href{https://profootballtalk.nbcsports.com/2019/04/28/jim-irsay-wants-the-draft-in-indianapolis/}{says he wants the draft in Indianapolis}.

Indianapolis would be a good spot except for one thing. It’s far enough north to make weather a problem. 

When Radio City Music Hall left the draft four years ago, Chicago lobbied hard to be the new permanent home. But anyone who looked out the window at the heavy snow coming down on Saturday, the last day of the draft, had to agree that the the NFL made the right decision when they started rotating cities instead. 

Indianapolis isn't that far south of Chicago.

\subsection*{One Final Thought}


\item Biggs \href{https://www.chicagotribune.com/sports/football/bears/ct-spt-bears-mailbag-stephen-denmark-david-montgomery-biggs-20190501-story.html}{continues to answer your questions}:

\begin{quote}
  ``The Bears are planning to sign 19 undrafted free agents. As you and others have documented, this has become an important part of the draft process as teams scramble and bid against one another to sign highly regarded prospects who slipped through the cracks. The draft was cut down to seven rounds in 1994. Do you think the NFL would consider adding more rounds given the importance of these undrafted players? — Tom S., Chicago

``No. Adding an eighth round would only force teams to have to pay players more. Good teams do well with undrafted free agents (UDFAs) and bad teams struggle to find players who can stick. Plus, isn’t the final day of the draft long enough?''
\end{quote}

Yes, if you are a reporter.  Maybe not if you are a general manager.

I actually didn't think this was a bad question.  Wisconsin guard Beau Benzschawel \href{https://twitter.com/RapSheet/status/1122584978724982784}{had offers from 20 teams} before signing with the Lions.  It's very evident that teams are leaving good players on the board at the end of seven rounds.

\end{itemize}
\end{document}