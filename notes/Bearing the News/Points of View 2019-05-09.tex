% Created 2019-02-17 Sun 09:59
% Intended LaTeX compiler: pdflatex
\documentclass[11pt]{article}
\usepackage[utf8]{inputenc}
\usepackage[T1]{fontenc}
\usepackage{graphicx}
\usepackage{grffile}
\usepackage{longtable}
\usepackage{wrapfig}
\usepackage{rotating}
\usepackage[normalem]{ulem}
\usepackage{amsmath}
\usepackage{textcomp}
\usepackage{amssymb}
\usepackage{capt-of}
\usepackage{hyperref}
\author{Tom Shannon}
\date{\today}
\title{}
\hypersetup{
 pdfauthor={Tom Shannon},
 pdftitle={},
 pdfkeywords={},
 pdfsubject={},
 pdfcreator={Emacs 26.1 (Org mode 9.1.9)}, 
 pdflang={English}}
\begin{document}

\subsection{Bears}
\begin{itemize}
\item  \textbf{Gregg Rosenthal} at \textit{NFL.com} \href{https://podcasts.apple.com/us/podcast/the-around-the-league-podcast/id680904259}{comments upon the Bears} during their ``Around the NFL'' podcast (episode title: ``Around the NFC'', at the 24:45 mark).

Rosenthal mentions his distrust of teams that stand pat after a successful season and expect that roster to come back and perform at the same level (at the 24:45 mark).  Not that the Bears had much choice, as he acknowledges.

I share Rosenthal's mistrust.  This didn't work in 2007 with the Bears coming off of a Super Bowl appearance and it probably won't work this year, at least for the defense.  As was the case in 2006, the Bears were among the healthiest teams in the league last year, something that is unlikely to happen a second time in a row.  They also lost most of the defensive coaching staff.

But the reasons for this wariness go deeper than that.  Last year with a new head coach the players were less comfortable and likely for the most part concentrated harder on what was going on.  This year with the stink of success on them, at least a certain percentage of the players are likely to be more relaxed.  This can lead to a drop in performance.

In fairness, unlike 2007, the Bears have reason to believe that their offense will be better in a second year in head coach \textbf{Matt Nagy}'s system.  This is their best hope to maintain excellence in the coming season.

\item \textbf{Mark Potash} at the \textit{Chicago Sun-Times} \href{https://chicago.suntimes.com/sports/first-and-10-bears-could-hit-the-jackpot-with-motivated-wr-emanuel-hall/}{quotes new Bears wide receiver \textbf{Emanuel Hall}} on his motivation after being passed over in the NFL draft.  Hall was expected to go as high as the second round:

\begin{quote}
  ``'I promise you it’s a feeling that I’ll never forget. That was one of the worst feelings ever, being undrafted,' Hall said at Bears rookie mini-camp Saturday. 'It felt like the longest three days of my life. I had a draft party on the second day and the third day you’re just shaking everybody’s hand, ‘Thanks for coming.’ — no one wants to do that.'''
  
``Hall’s speed (4.39 in the 40) and big-play ability made him an intriguing prospect in the draft. He averaged 23.5 yards per catch in his final two years at Missouri (70 receptions, 1,645 yards, 14 touchdowns in 22 games), with nine receptions of 50 yards or more.

``But a history of minor injuries that kept him out of four games last season and prevented him from playing in the Senior Bowl played a part in getting overlooked in the draft.''
\end{quote}

Hall fell out of the draft due to concerns about his football character.  Specifically, coaches at the University of Missouri were very up front about Hall's inability to play through those minor injuries that Potash mentions.  In fact, Hall missed the last day of rookie minicamp with an injury.  \textbf{Kevin Fishbain} at \textit{The Athletic} elaborates:

\begin{quote}
  ``Hall battled hamstring issues and a groin injury last season, but instead of shutting it down, returned to finish off the season and play in the bowl game. Even though he got back on the field, the injury required maintenance in the winter, preventing him from playing in the Senior Bowl, and it almost kept him from performing at the combine.

  ``Doctors in Indianapolis told Hall not to participate in drills because of a possible sports hernia. Hall signed a waiver so he could run and jump, and while not at 100 percent, he ran a 4.39 40 and had the best broad jump recorded for a wide receiver at the combine (11 feet, nine inches).''

  ``Nagy discussed, though, how the Bears are confident in their staff to help with players who may have had durability issues in college.
  
``'There’s some elements to that in regards to our training program is going to be different than every other training program from other teams that they come from,' he said. 'We feel really good about \textbf{Andre Tucker} and what he does and our staff that he has, our strength staff, \textbf{Jenn (Gibson)} our sports dietician. And so we feel really good about when we bring people in here, we really sometimes don’t care as much about what happened in the past — what can we do now? Let’s fix it, let’s give him a clean slate and let’s roll.”
\end{quote}

Speed and ability aren't enough.  Apparently Hall is going to have to toughen up.  Otherwise he'll be just another track star that didn't make it.
\item \textbf{Brad Biggs} at the \textit{Chicago Tribune} \href{https://www.chicagotribune.com/sports/football/bears/ct-spt-bears-mailbag-receivers-kicker-competition-20190508-story.html}{answers your questions}:

\begin{quote}
  ``Do you see the Bears signing any veteran pass rushers now that it won’t impact their comp picks? If yes, who? — @mellothunder

``This has been a consistent question throughout the offseason, even after the team re-signed veteran outside linebacker \textbf{Aaron Lynch}, and I’m a little puzzled why. The Bears tied for third in the league with 50 sacks last season. Only three of those 50 sacks were by players no longer on the team: Nickel cornerback \textbf{Bryce Callahan} had two and strong safety \textbf{Adrian Amos} had one. With \textbf{Khalil Mack} being paid as one of the top edge rushers in the game and with investments elsewhere when it comes to rushing the passer, I think the Bears are OK here. I can’t see them spending a lot of money for another player to add to the mix.''
\end{quote}

I think I understand why.  On some level fans probably recognize that the Bears depth is likely to be challenged more this year and they are worried about it with only Mack, \textbf{Leonard Floyd} and Lynch as reliable pass rushers.

Personally I'm not as worried about ti for two reasons.

\begin{enumerate}
\item Not many teams have good starting pass rushers, let alone depth at th position.  Good pass rushers don't grow on trees and anyone who is any good was signed to a roster a long time ago, compensatory formula or not.
\item I like \textbf{Kylie Fits} and \textbf{Isaiah Irving} a lot more than most people and think one or both could step up and do a decent job if called upon.
\end{enumerate}

It will be interesting to see how the depth at the position plays out this year.
\item \href{https://www.chicagotribune.com/sports/football/bears/ct-spt-bears-mailbag-receivers-kicker-competition-20190508-story.html}{Another good question} for Biggs:

\begin{quote}
``Should the Bears pursue \textbf{Ndamukong Suh}? I think that would be a great addition. — @chiwest773

Teams can now sign free agents such as Suh without that transaction affecting the formula for compensatory draft picks, so we could begin to see some movement for some of the name players who remain on the street. Suh is in that category, as is offensive tackle \textbf{Jared Veldheer}, who is reported to be signing with the Patriots. I don’t believe the Bears are planning to spend big money on any players not on the roster. It’s possible offensive lineman \textbf{Cody Whitehair}, who’s eligible for a contract extension, is the only player who will get a significant payday between now and the end of the season.
\end{quote}

Suh would actually be a good fit for the Bears strictly in terms of what he can still do on the field.  At this point in his career he can still be a big, two gapping defensive lineman who will stop the run while giving a little bit of pass rush.  But he's nowhere near the player he was earlier in his career the last time Bears fans saw him on a regular basis with the Lions.  Suh is 32 years old and the Bears have trended towards younger free agents since general manager \textbf{Ryan Pace} joined the organization.  He's also not a great culture fit.

Whether Suh signs with the Bears or anyone else will come down to money.  My guess is that no one is going to make him a big money offer at this point.  As Biggs points out, the Bears almost certainly won't.  A lot will depend on whether Suh still wants to play the game for considerably less than he's used to making.
\item \href{https://www.chicagotribune.com/sports/football/bears/ct-spt-bears-mailbag-receivers-kicker-competition-20190508-story.html}{Yet another question} for Biggs:

\begin{quote}
Is \textbf{Brad Childress} under contract for the full season or just the preseason like last year? What role will he play with the offense? — \@\_d\_r\_r\_

The Bears hired Childress as a senior offensive assistant, and he will be around for the entire season. He was hired as a consultant last year, working with Matt Nagy from the start of his tenure and through most of the preseason. He’ll be available as a sounding board for Nagy and will have input across the board on offense. Nagy is big on taking input from all of his coaches, and Childress will be part of that mix.
\end{quote}

If you look back at the history of this blog, you'll find that I had some rough things to say about Childress as a head coach for the Vikings.  But I like him in this role.  He's from Chicago and at this point in his career he probably wants to be here.  He's experienced in a way that Nagy isn't and there's very little doubt in my mind that he'll say what he thinks at times when maybe other assistants will hold back a little.  

But there's one other under looked factor here that could be of great benefit to Nagy and the Bears.  Childress has traditionally been a ``run first'' offensive coach.  He knows how to run the ball and that knowledge could be handy on a team where Nagy has constantly said that the running game has to get better.  Childress could help contribute to a big improvement in that area.
\item \href{https://www.chicagotribune.com/sports/football/bears/ct-spt-bears-mailbag-receivers-kicker-competition-20190508-story.html}{One more} from Biggs:

\begin{quote}
Reading about all the early draftee signings, I remember a time when \textbf{Cliff Stein} was first to get the Bears draft class under contract. Lately it seems they’ve been bringing up the rear. What happened? — Greg M.

What’s the race? The NFL has slotted the bonus money and salaries for draft picks. There’s no advantage to completing this right away. I have no doubt the Bears will have all five draft picks under contract before training camp begins.
\end{quote}

I'm pretty sure the agent was mostly to blame for the Bears troubles signing first round pick \textbf{Roquon Smith} last year.  Nevertheless, one does wonder if the Bears could have settled the issues with him sooner had they been further along with negotiations when camp started.

Biggs asks ``What's the race?'' and that's fair.  But my question is, ``Why wait?''.  Get the rookies under contract so that they can workout without fear of injury.

In any case, with their first pick coming in the third round, I doubt the Bears will have a great deal of trouble signing their picks before camp this year.
\item Fishbain \href{https://theathletic.com/962002/2019/05/04/year-3-your-comfort-level-with-everything-just-goes-up-qa-with-bears-quarterback-mitch-trubisky/}{interviews \textbf{Mitch Trubisky}}.  He asks Trubisky about the center position:

\begin{quote}
Q: ``Gotta build up that friendship with \textbf{James} [\textbf{Danels}] now.''

A: ``Ha, yeah, got to. James knows. I’m talking to James even more. He’s like, 'Why is Mitch talking to me all the time?' I was like, 'You’re my center now, bro, let’s do it.'''
\end{quote}

The Bears haven't announced it, yet, but this apparently let's the catch out of the bag.  I gather from this that \textbf{Cody Whitehair} is switching positions with Daniels with Whitehair moving to left guard.

It's a risky switch.  Daniels will undoubtedly be better at center but Whitehair was a Pro Bowl caliber center who was still getting better.  Whether the unit overall is be better with this switch will be one of the more interesting questions early in the season.
\end{itemize}

\subsection{One Final Thought}
Although I did \href{http://bearingthenews.com/blog/2019/05/02/should-the-nfl-add-another-round-to-the-draft-and-other-points-of-view/}{threaten to get sick} if I read another kicker article, I have to admit to letting go of a chuckle when \textbf{Darin Gantt} \textit{profootballtalk.com} \href{https://profootballtalk.nbcsports.com/2019/05/03/bears-bringing-eight-kickers-to-rookie-minicamp/}{characterized what the Bears are doing at the position as a ``clown-car approach''}.

\end{document}