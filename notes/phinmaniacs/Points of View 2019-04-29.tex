\begin{itemize}
\item Omar Kelly at the Sun-Sentinel \href{https://www.sun-sentinel.com/sports/miami-dolphins/fl-sp-dolphins-chris-grier-20190426-story.html}{criticizes the Dolphins trade for new Dolphins quarterback Josh Rosen}:

\begin{quote}
  ``[T]he Cardinals fleeced the Dolphins for a quarterback nobody else in the NFL seemingly wanted.''

  ``Despite having no other bidders for Rosen’s services, the Dolphins not only sent Arizona a second-round pick for the UCLA product. They also sweetened the deal by trading away a 2020 fifth-round pick.''
\end{quote}

That's not the whole story by a long shot.  Any fair evaluation of this trade has to include the previous exchange where the Dolphins traded back from the 48th overall pick in the second round to the 62nd.  Here's how the trade breaks down assuming, as is the common wisdom, that future picks are worth a pick in the current year minus 1 round:

\begin{tabular}{lllll}
  \hline
  Trade &Dolphins give &2019 equivalent &Dolphins get &2019 equivalent\\\hline
  Dolphins trade back with Saints from 48 overall to 62 overall &second rounder & &second rounder &\\
        &fourth rounder &&second rounder (2020) &third rounder\\
  &&sixth rounder &\\
  Dolphins trade 62 overall to Cardinals for Josh Rosen &second rounder & &Josh Rosen &\\
        &fifth rounder (2020) &sixth rounder &&\\\hline
\end{tabular}
  Add all of that up and here's what that means.  After all is said and sifted, the Dolphins got a top ten pick from 2018 and a potential starter for trading back to the third round and giving a fourth round pick.  Can anyone honestly tell me that if the Dolphins had made that trade while sitting at 48 overall that they shouldn't have taken it?

Ladies and gentlemen, that is a deal.  The Dolphins not only didn't get fleeced, they took the Cardinals to the cleaners.

  The Dolphins did the right thing here.  They waited the Cardinals out and bargained hard and got their guy.  And they did it in such a way that if Rosen doesn't work out, they can let him go and figure that they didn't lose that much.  So those who want a quarterback in 2020 aren't out anything by this deal.

  The only real question is whether one year will be enough to properly evaluate Rosen.  It certainly will take longer than that to fully develop him.  But it says here that if the Dolphins know what they're doing, they'll at least have a good idea of what they have by the time the 2020 draft rolls around.

  And from what I can tell so far, they do know what they're doing.
\item Phil Rosenthal and Tim Bannon at the Chicago Tribune \href{https://www.chicagotribune.com/sports/football/bears/ct-spt-bears-nfl-draft-winners-losers-espn-20190429-story.html}{cover the draft's winners and losers and don't while not being stupid about it}.  Dolphins fans will want to skip to the last few.

  In the mean time I'd like to add my own winner:  \href{https://en.wikipedia.org/wiki/Joel_Klatt}{Joe Klatt} at the NFL Network.  Klatt usually covers college football for FOX but he came on and did a great job during Day 2 of the draft breaking down the picks.  I spent most of Day 3 wishing he was still there instead of \href{https://en.wikipedia.org/wiki/Peter_Schrager}{Peter Schrager}, who seemed to be there more for the entertainment value.
  
\item Safid Deen at the Sun-Sentinel \href{https://www.sun-sentinel.com/sports/miami-dolphins/fl-sp-dolphins-undrafted-rookies-20190427-story.html}{lists the known undrafted free agent signings} for the Dolphins.

I'm not going to go down this list and pretend I know anything about these guys because for the most part I don't.  But I will tell you that when they hit the field this summer, you might want to pay attention.

The Dolphins have a lot of roster openings and a lot of holes to fill.  Some of them are gong to be filled by these signings.  

And I'll add this: if you want to evaluate Chris Grier as a GM, paying attention to how many of these undrafted free agents develop is one good way to do it.  Good teams with good front offices and coaching staffs usually find a way to develop a few of these kinds of players into good, solid starters.  And these guys are going to have more than the usual opportunity to show what they can do.

\item Scott Bordow at the Arizona Republic \href{https://theathletic.com/945667/2019/04/25/2019-nfl-draft-live-tracker-dane-brugler/}{reviews the pick of quarterback Kyler Murray} for The Athletic:

\begin{quote}
  Cardinals general manager Steve Keim said he didn’t want to take Kyler Murray after trading up to get Josh Rosen in the first round last year but he was won over by Murray’s talent. This is a gamble for Arizona. Few teams had Murray as the No. 1 player in the draft and it’s fair to wonder if the Cardinals would even have considered Murray if they hired anyone other than Kliff Kingsbury as coach. This will either work out spectacularly for Arizona or cost Keim and Kingsbury their jobs some day.
\end{quote}

My guess is that Bordow is right.  The Cardinals don't take Murray if anyone other than Kingsbury is the head coach.  The reason is simple.  Murray fits what the Cardinals want to do out of the box whereas any other coach outside of Seattle would have had to change their entire offense to make Murray work.  Murray's size makes him less than suitable for an offense that relies on sitting in the pocket and finding the open receiver.

I tried to figure out where the next likely landing place for Murray was if he got by the Cardinals.  My guess is that it would have been a long way down the list.  Murray not only landed in the perfect spot to take advantage of his talents.  He may have landed in the only spot.  
\end{itemize}
