% Created 2019-02-17 Sun 09:59
% Intended LaTeX compiler: pdflatex
\documentclass[11pt]{article}
\usepackage[utf8]{inputenc}
\usepackage[T1]{fontenc}
\usepackage{graphicx}
\usepackage{grffile}
\usepackage{longtable}
\usepackage{wrapfig}
\usepackage{rotating}
\usepackage[normalem]{ulem}
\usepackage{amsmath}
\usepackage{textcomp}
\usepackage{amssymb}
\usepackage{capt-of}
\usepackage{hyperref}
\author{Tom Shannon}
\date{\today}
\title{}
\hypersetup{
 pdfauthor={Tom Shannon},
 pdftitle={},
 pdfkeywords={},
 pdfsubject={},
 pdfcreator={Emacs 26.1 (Org mode 9.1.9)}, 
 pdflang={English}}
\begin{document}


The Sun-Sentinel does \href{https://www.sun-sentinel.com/sports/miami-dolphins/sfl-star-nfl-qbs-whose-first-years-have-been-inferior-to-the-rest-of-their-careers-20190503-photogallery.html}{a photo gallery} of quarterbacks who, similar to new Dolphins quarterback Josh Rosen, had slow starts to their careers.  In one photo they highlighted the start of Peyton Manning's career:

\begin{quote}
  ``His first game was against the visiting Dolphins who grabbed three of his passes, including one for a pick-six by Terrell Buckley in a 24-15 Miami win. By the time the Colts had finished the first quarter of the season, Manning had piled up 11 pickoffs against only three touchdown passes. Things stabilized from there as he threw 23 touchdowns and 17 interceptions in the final 12 games. The rest is Hall of Fame history.''
\end{quote}

I hate clicking through these photo galleries and I usually don't bother with them.  But I knew I was going to have to read this one because I knew Manning would come up on a list of very good quarterbacks.  With all due respect to Tom Brady, Dan Marino and the many others you could mention, Manning is in my opinion the greatest quarterback in the history of the game.  I love all of those other guys but not one of them had to literally lift their team and carry them the way Manning did.

But that's not my point.  My point is that Manning elevated his game from his first \href{https://en.wikipedia.org/wiki/1998_Indianapolis_Colts_season}{3 win season} through to NFL history with the help of very good coaching from legendary offensive coordinator Jim Moore and quarterbacks coach Bruce Arians.  And Rosen is going to need that kind of help.

Like Manning, Rosen is smart.  In fact, \href{https://www.sun-sentinel.com/sports/dave-hyde-blog/fl-sp-hyde5-jimmy-johnson-rosen-20190503-story.html}{in the words of former Dolphins head coach Jimmy Johnson}, probably too smart.

What did Johnson mean?

Rosen has a reputation for being arrogant.  Most Dolphins fans can't figure out where that reputation came from because \href{https://www.sun-sentinel.com/sports/miami-dolphins/fl-sp-hyde-dolphins-rosen-20190429-story.html}{his introduction to the Miami media, as least, was nothing but positive}.  But I have an idea how Rosen might have gotten stuck with this label.

In my day job when I'm not bloviating about the NFL, I teach first year medical students.  Specifically, I deal with a lot of extremely intelligent, high achieving students that I greatly respect.  And those students challenge me by with a lot of questions.  In fact, even after doing this for going on 17 years, I'm constantly amazed at my students' ability to come up with things I've never heard before.

Sometimes these interactions test my limit when trying to show that I know what I'm talking about.  And, at least as important, they test my willingness to admit when I don't know what I'm talking about.  I enjoy these conversations, especially when the latter is true, because it forces me to learn something.  I'm nearly always better for having talked to a smart student about a topic.

But I have tell you honestly that not all of my colleagues feel the same.  There are some who believe that its their job to always show students that they are superior and they don't like interactions with students who, frankly, might be smarter, if less knowledgeable, than they are.  In the end, they explain their shortcomings to themselves by blaming the students.  Most feel that these students lack respect.

And that brings us back to Josh Rosen, who is by all accounts an extremely intelligent player.  And, let's be honest, his teachers haven't been the best.

After having had three offensive coordinators in college at UCLA, Rosen is going on his third offensive coordinator as he joins the Dolphins this season after only one year with the Cardinals.  Up until this point, \href{http://www.espn.com/blog/nflnation/post/_/id/288438/the-good-and-bad-of-josh-rosens-five-offensive-coordinators-in-four-years}{Rosen has had to deal with a carnival of coaches} including Mike McCoy, who was fired after week 7 last year, followed by Byron Leftwich, who was fired after the season.  

Someone with more cruelty in his blood than I have than me might have called it a "clown car".  

Nothing against the Cardinals staff but you don't go through coaches like that if you are good at what you are doing.

Bottom line, it would be no surprise to find that Rosen was smarter than his teachers.  And it would be no surprise to find that his teachers didn't react well to the situation.

So far it seems that Rosen has had the same benefits that Manning had in his first seasons with the Colts.  Let's hope that he finds a better home in South Florida with good instructors that he can respect and, just as important, who can respect him.
\end{document}
