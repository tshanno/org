% Created 2019-02-17 Sun 09:59
% Intended LaTeX compiler: pdflatex
\documentclass[11pt]{article}
\usepackage[utf8]{inputenc}
\usepackage[T1]{fontenc}
\usepackage{graphicx}
\usepackage{grffile}
\usepackage{longtable}
\usepackage{wrapfig}
\usepackage{rotating}
\usepackage[normalem]{ulem}
\usepackage{amsmath}
\usepackage{textcomp}
\usepackage{amssymb}
\usepackage{capt-of}
\usepackage{hyperref}
\author{Tom Shannon}
\date{\today}
\title{}
\hypersetup{
 pdfauthor={Tom Shannon},
 pdftitle={},
 pdfkeywords={},
 pdfsubject={},
 pdfcreator={Emacs 26.1 (Org mode 9.1.9)}, 
 pdflang={English}}
\begin{document}

\begin{itemize}
\item Armando Salguero \href{https://www.miamiherald.com/sports/spt-columns-blogs/armando-salguero/article229939589.html}{separates spin vs. fact about the Dolphins draft}:

\begin{quote}
  ``Narrative: The Dolphins trade down in the second round was part of Miami’s plan to land Rosen.

``Mando verdict: Not only is this spin, it is demonstrably untrue.

``We understand that from general manager Chris Grier himself. From the Dolphins’ news conference at the end of the draft Saturday:

``Question: Were there discussions at pick 48? Did you have to trade back from 48 to make that [Rosen] trade happen in your mind, to make the numbers add up?

``Answer: 'No. The Cardinals were steadfast in what they wanted for it and they were pretty up front with all of the teams. We hadn’t really talked to them until right at the start of the draft, and I think that’s when they started contacting everyone that might be involved. As we went through, we had talked throughout the day a little bit here and there and we finally got to a point where we were comfortable in making a pick. In terms of picking up, for us, it was huge to get the second-round pick in 2020, with the Saints when we made that trade. Regardless of what was there, we were going to make that trade to get the second-round pick. We went into the draft trying to find either another first or second-round pick in 2020.'

``In one-on-one phone interviews served up by the Dolphins media relations department for a couple of national writers, Grier told both Monday Morning quarterback and Football Morning in America that the two were not connected. And he told FMIA that reports of a deal with Arizona being done earlier that somehow guaranteed a trade for Rosen after Miami traded down were wrong.''
\end{quote}

Salguero makes a lot of good points in this article but that one falls flat.

Salguero's point is that the Dolphins made the trade not knowing if they could swing a deal for Rosen with the 62nd overall pick or not.  No deal was in place.  I have absolutely no trouble believing that.

At the same time, Salguero, perhaps purposely, misses the point.  The fact that the Dolphins felt good making the trade even if Rosen didn't come with it doesn't mean that they didn't have a potential Rosen deal in mind when they made it.  In fact, given that they were already in discussions and thought the 48th pick was too high but obviously felt the 62nd wouldn't be, it would be foolish not to assume that they didn't.

Whether the Dolphins were willing to stand on this trade without a Rosen deal or not, the end result is the same.  They basically \href{https://www.phinmaniacs.com/news/miami-dolphins-post-draft-points-of-view}{traded back into the third round and added a fourth round pick to make this happen}.  Intent aside, that was the end result of their maneuvering and that is the bottom line.

Any other conclusion is just spin.

\item \href{https://www.phinmaniacs.com/news/its-okay-to-change-your-mind-about-dolphins-qb-josh-rosen}{What if Ryan Fitzpatrick wins the quarterback competition over Josh Rosen?}

"If he wins the competition, absolutely \href{https://www.sun-sentinel.com/sports/miami-dolphins/fl-sp-dolphins-qb-competition-20190510-dj3du7vsznaddn36ju64poo73m-story.html}{I’m good with that}. I think that would be what’s best for the team and best for the Miami Dolphins," Dolphins coach Brian Flores said.

Right.

This is, of course, what Flores has to say.  But this competition is - or had better be - slanted towards finding out what the young, potential franchise quarterback can do.  

It's absolutely true that you can't just trot Rosen out there no matter what he does.  For one thing, evaluating Rosen isn't the coaching staff's only job.  Every young player on the team has to be examined and a determination has to be made about their future with the franchise.  That's particularly true of Kenyan Drake and Jakeem Grant, \href{https://www.sun-sentinel.com/sports/miami-dolphins/fl-sp-dolphins-xavien-howard-20190510-nmebckxrcnfnpfp2pamje77aeu-story.html}{both of whom are entering contract years}.

How can you evaluate your talent if the quarterback if the quarterback can't throw the ball accurately or run the offense?  How can you develop younger players?

So you do have to have a competent quarterback in order to develop and run the team properly in a rebuilding year.  But having said that if Fitzpatrick wins this job it will mean very bad things for Rosen.  And it will mean the decision about what to do come draft time in 2020 will have already been made.


\item Omar Kelly at the Sun-Sentinel href{https://www.sun-sentinel.com/sports/miami-dolphins/fl-sp-dolphins-sidebar-20190509-gpapecygtve4vnitzeveudpkne-story.html}{describes the defensive scheme} that the Dolphins plan to run:

\begin{quote}
``Miami’s coaches intend to run a hybrid scheme that incorporates both elements of a 4-3 and 3-4 front, and envision themselves playing with five defensive backs on the field at the same time, possibly as the team’s base defense.

``Charles Harris, the team’s 2017 first-round pick, doesn’t have a clear cut position. Is he a defensive end or a linebacker?

``Minkah Fitzpatrick, Miami’s 2018 first-round pick, doesn’t have a defined role. Will he be playing free safety, nickel cornerback, a cornerback on the boundary, or all of the above?''
\end{quote}

I think Kelly describes the forest well but he misses the point when it comes to the trees.  The players in this defense will be expected to do a lot of things but to say that they don't have ``a defined role'' really isn't true.  In fact, if the coaches do their jobs right, their role will be very clear cut and well-defined in any given situation.  That won't be - or at least it shouldn't be - the problem.

The problem comes in when you consider the talent of your players and whether they are capable of executing those well-defined roles.  In this case, new Dolphins defensive line coach Marion Hobby seems to understand the challenge.

  ``That’s where the awareness comes in,'' Hobby said.

  ``A coach used to always tell me if you trick the [defensive] ends and trick the free safety, you’re going to get a big play.  So those guys have to have some awareness to them. They have to play with their eyes and their feet. It’s hard. There are very few that can.''

  When you are playing a multiple 3-4, 4-3 scheme, to a certain extent you throw the old position definitions away.  But position definitions are still there.  They're just new.  The trick, just as it is in the old standard systems, is the same.  Getting guys who are capable of doing their jobs and getting them to the point where they can execute them is still the key to successful football.
\item Salguero \href{https://www.miamiherald.com/sports/spt-columns-blogs/armando-salguero/article230246299.html}{stumps for a Laremy Tunsil contract extension}:

\begin{quote}
``Look, Tunsil has not fully arrived. He’s not fully developed. There’s much room for growth. But no one in the Dolphins organization questions whether he should be part of the organization going forward. Because he’s very good.

``So, I’m told, there are plans to get Tunsil locked up long term also -- perhaps before the start of the regular season.''

``Paying now will ultimately be cheaper than paying later. Trust me, player salaries rarely go down. So signing Tunsil to an extension would be beneficial for 2020 and ‘21 cap purposes.''
\end{quote}

Salguero has a point.  But there are drawbacks.

As Salguero points out, the Dolphins don’t have to do this. Tunsil is under contract for 2019 and they have already picked up the fifth-year option for 2020.  If they had to, they could apply the franchise tag in 2021.

Interestingly, Omar Kelly at the Sun-Sentinel \href{https://www.sun-sentinel.com/sports/miami-dolphins/fl-sp-dolphins-xavien-howard-20190510-nmebckxrcnfnpfp2pamje77aeu-story.html}{asserts out in another article} that the Dolphins have been criticized for trying to do contract extensions too late.  He points to losses like defensive end Olivier Vernon, tight end Charles Clay, receiver Jarvis Landry, tailback Lamar Miller and offensive tackle Ja’Wuan James under Mike Tannenbaum.  I disagree with this assessment.  In each of these cases it wasn't a question of approaching the player too late.  It was a question of money and value where, right or wrong, the Dolphins in each case decided that there wasn't a match.

In fact, I would argue the opposite.  The Dolphins got themselves in some trouble after Adam Gase's first year by handing out a lot of contracts such as those for Andre Branch, Rashad Jones, and Kiko Alonzo that they'd probably like to get out from under.  In many cases, these players were extended when the Dolphins really didn't have to do it and they eventually let the team down either through disappointing play (debatably Alonzo), a lack of development (Branch) or poor football character (Jones).

Bottom line, a lot of money was spent that didn't have to be and probably wouldn't have been had the teams waited.

There are also some additional factors.  Players who get early long term deals tend to get comfortable.  This can lead to less concentration and a decline in quality of play.

And there is the fact that this sort of thing leads to the expectation on the part of other players that their contracts will be extended early, too.  So even if you aren't worried about extending Tunsil, the next guy who is more borderline will press for an early extension.  And when he doesn't get it, it can lead to an early hold out, one which agents have reason to expect might work simply because the team has a history of giving early extensions.

On balance, I'd say its better to wait until the 2020 offseason to extend Tunsil.  It would prevent Tunsil from having to enter a final, lame duck year and keep him out of the free agent market.  This while keeping the team from having to tag him and while setting a better pattern for other players under the new regime.  In the meantime, even if you are reasonably sure Tunsil is a part of your future, it gives you another year to make sure Tunsil remains healthy and to see how he develops and performs under the new coaching staff.

\item Safid Deen at the Sun-Sentinel reports that the Dolphins \href{Safid Dee}{are giving running back Mark Walton an try out}.

I found this move to be ironic given that on the very same day Flores \href{https://www.sun-sentinel.com/sports/miami-dolphins/fl-sp-dolphins-xavien-howard-20190510-nmebckxrcnfnpfp2pamje77aeu-story.html}{characterized the extension of Xavien Howard's contract as a move meant to build team culture}.

Walton was cut by the Cincinnati Bengals last month after his third arrest.  He is facing felony charges for allegedly carrying a concealed weapon, marijuana possession and reckless driving. Significantly, the March 12 incident took place in Miami, his home town.  Now the Dolphins are not only giving him a second chance, they've brought him back to a city where old friends and bad influences could exacerbate the problem.

But that's not my point.  The problem is that they've brought him into a young locker room where veteran influence will have more than the usual impact on the development of the team.

I don't have a problem with Walton getting a second chance.  But giving it to him with the Dolphins seems like a very questionable move.
\end{itemize}
\end{document}