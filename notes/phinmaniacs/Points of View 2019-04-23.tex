\begin{itemize}
\item Dave Hyde at the Sun-Sentinel \href{https://www.sun-sentinel.com/sports/miami-dolphins/fl-sp-hyde-dolphins-grier-20190420-story.html}{comments on something that I think has had us all a bit puzzled}.

"By far, the most common question about the Dolphins after the rhetorical tank-of-no-tank (seriously?) is this: Why was Chris Grier promoted to king-of-his-world powers when his drafts seem mediocre the past three years?"

I think the answer that you;ll get from most fans is that its because Mike Tannenbaum was defacto general manager and Adam Gase had too much influence over personnel.  And they'd have a point.  But only to an extent.

I have little trouble that Tannenbaum and Gase had strong opinions about who the first and second round picks should be.  But after that, more and more it becomes scouts work.  And that's Chris Grier's domain.

So how has Grier done with those crucial mid- and late-round picks?  Hard to say.  He's only presided over 3 drafts but let's take a look.  Also note the results of an informal poll amongst out Phinmaniacs writers (1=poor, 5= excellent).  (Yes, I know its stupid to grade recent drafts.  But for the sake of the exercise, let's do it anyway.)

2016:
3rd round:  Kenyan Drake
3rd round:  Leonte Carroo
6th round:  Jakeem Grant
6th round:  Jordan Lucas
7th round:  Brandon Doughty
7th round:  Thomas Duarte

Phinmaniacs writers:  3.0 out of 5

This was, indeed, not great.  Drake has been a solid contributor but they needed him to develop into the main running back.  He had the opportunity last year but it seemed that no one associated with the team was beating the drum to put him in that position, reportedly due to some maturity issues.  Carroo has been waived after what was a less than productive career but Grant was a hit.  I think a good draft would have had at least a solid contributor in at least one of those last three names.

2017:
3rd round:  Cordrea Tankersly
5th round:  Isaac Asiata
5th round:  David Godchaux
6th round:  Vincent Taylor
7th round:  Isaiah Ford

Phinmaniacs writers:  3.0/5

Really hard to tell only two years out from this draft.  Tankersly has been a disappointment when called upon to play.  I'd call both Godchaux and Taylor hits as they would both be solid members of the rotation on the defensive line this year without the scheme change.  It wouldn't have been a good defensive line but they're still both starters.  

A lot will depend on how Asiata turns out.  At the moment, he hasn't impressed anyone in limited exposure but he'll likely get his chance to compete to start this year.  

2018:
3rd round:  Jerome Baker
4th round:  Durham Smythe
6th round:  Kalen Ballage
6th round:  Cornell Armstrong
7th round:  Quentin Poling
7th round:  Jason Sanders

Phinmaniacs writers:  3.8/5

Like our writers I feel better about this class than the previous two.  Again, its still early but already pretty good.  Baker and Sanders are already solid hits and Ballage shows some signs of developing into one. It's discouraging how little the others have contributed but I think its hard to ask for more than three good hits out of six in these rounds.

My conclusion is that Hyde's characterization of Grier isn't unfair.  But I also think the fans have a point.  

To have really called those picks excellent, I would have liked to have seen just one more good hit in each of those years above.  But its fair to say that if you give Grier a free hand on those early round picks, his record in the later round should give fans some signs of hope.

\item Marc Sessler at NFL.com \href{http://www.nfl.com/news/story/0ap3000001027184/article/nfl-draft-ranking-every-quarterback-class-of-this-millennium?campaign=Twitter_atn}{ranks the quarterback draft classes of the millennium}.  The 2017 class came in at number 7 and, as all three quarterbacks have become Pro Bowlers with in just two years, it arguably should have been higher.  

  Why is that significant to Dolphins fans?  Because that class \href{https://www.chicagotribune.com/sports/football/bears/ct-spt-bears-pro-bowl-eddie-jackson-mitch-trubisky-patrick-mahomes-20190125-story.html}{was denigrated} at least as much as the 2019 class is currently being criticized.

  \begin{quote}
    “So many people said it wasn’t a strong quarterbacks class,” [Mitch] Trubisky said.

Added [DeShaun] Watson: “We all remember that. Patrick [Mahommes] has said it. Mitch said it. I’ve said it. And we all put that in the back of our heads, went to work and started grinding. … The whole time leading up to that draft, no one thought we would be where we’re at now. Especially after Year 2.”
  \end{quote}


Those opinions caused both Watson and reigning MVP Mahommes to fall into a range where teams that know what they're doing could trade up and get them while teams that don't know what they're doing took a pass.

It just goes to show how little the prevailing media opinion should influence fans' attitudes toward the draft.  If you are praying that the Dolphins don't take a quarterback because "all of the potential first rounders would have ranked behind all of last year's first rounders", you should remember: it's all been said before.

\item Armando Salguero at the Miami Herald thinks that the quarterback class is ``good'' but wonders if the Dolphins will take the gamble.

  \begin{quote}
    uddenly the mock drafts that weeks ago had Miami picking a quarterback in the first round have changed. The pundits and analysts have moved on to seemingly other ideas. The idea of the Dolphins picking a quarterback in the first round in 2019 seems less popular now.

And I don’t know why.

Because no one within the organization has dismissed the idea. The possibility still exists.

But it is right that the chances don’t seem huge.

So why the apparent shift?

Well, I think what we have is not a change in the Dolphins but a change in the so-called analyzing of the team. I think the analysis has caught up with where the team has probably been all along.

And that is, the Dolphins are open to picking a quarterback in the first round this year. But it has to be exactly the right quarterback. It has to be someone they’re truly convinced will take the team into the next decade.

And if that guy is not found and available, the team will be perfectly content passing on a first-round quarterback and aiming for that franchise guy later — like even in the 2020 draft.
\end{quote}

Fair enough.  But here's the problem.  Teams that have a need at quarterback, or any other position, have a bad habit of finding that they like the players.

You hear ex-NFL personnell men talk about this all the time.  You always say you'll take the best available.  But somehow the best available usually ends up being at a position of need.  You pay more attention to those prospects and they tend to get pushed up your board.

So, yes, absolutely.  The Dolphins shouldn't take a quarterback just to take one.  They have to believe in him.  But don't be at all surprised if they find that they believe in one of the ones that are right in front of them.

\item Assuming that the Dolphins don't go quarterback in the first couple rounds, I absolutely agree with what I've heard and read from most fans and media members - that building the line of scrimmage is the way to go in this draft.

  Having said that, the pass rushers have gotten the most first round attention - with some justification.  But dont' sleep on the defensive tackles.  Experts are calling this class of interior defensive linement one of the best ever.

  Its not as flashy as the sack generators - and the Dolphins could certainly use some with Charles Harris being their best option right now.  But nose tackle is a big deal is you are going to spend any time in a three man front as the Dolphins are apt to do next year.  There would be nothing wrong with drafting a big man who can move, either there or at defensive end.

  
\item  Just a quick note as we wrap up the pre-draft phase of free agency.  I understand why many will disagree but \href{https://www.sun-sentinel.com/sports/miami-dolphins/fl-sp-dolphins-eric-rowe-20190313-story.html}{Eric Rowe could be the Dolphins best signing of the period}.

  New head coach Brian Flores should be intimatley familiar with Rowe and he must have liked what he saw.  He was a depth piece in New England but its possible tht Flores thought he could be more.

  Rowe is only 26 years old and is far from his peak.  Identifying and developing young players who are on the rise is what the game is all about right now.  Rowe could be a very nice, affordable answer on a team that is re-building and that is about to sign Pro Bowl cornerback Xavien Howard on the other side to a lucrative extension.
\end{itemize}
