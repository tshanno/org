% Created 2019-02-17 Sun 09:59
% Intended LaTeX compiler: pdflatex
\documentclass[11pt]{article}
\usepackage[utf8]{inputenc}
\usepackage[T1]{fontenc}
\usepackage{graphicx}
\usepackage{grffile}
\usepackage{longtable}
\usepackage{wrapfig}
\usepackage{rotating}
\usepackage[normalem]{ulem}
\usepackage{amsmath}
\usepackage{textcomp}
\usepackage{amssymb}
\usepackage{capt-of}
\usepackage{hyperref}
\author{Tom Shannon}
\date{\today}
\title{}
\hypersetup{
 pdfauthor={Tom Shannon},
 pdftitle={},
 pdfkeywords={},
 pdfsubject={},
 pdfcreator={Emacs 26.1 (Org mode 9.1.9)}, 
 pdflang={English}}
\begin{document}

\begin{itemize}
\item Armando Salguero at the Miami Herald \href{https://www.miamiherald.com/sports/spt-columns-blogs/armando-salguero/article230422099.html}{writes about the definition of leadership by Dolphins head coach Brian Flores}: 

\begin{quote}
``'If you work hard and put the team first, you’re a leader,' Flores said. 'I want to have 53 leaders on our team. I want 90 on our team right now. That’s something that you can develop. That’s something that you can talk about.'''

``Flores is right to point out that players on other teams might stay away and be leaders.''

``But everyone in that room Tuesday knew I was asking about the Dolphins. And [Rashad] Jones.

``Is he a leader on the Dolphins?''
\end{quote}

I'm glad that Salguero is pressing the issue of Rashad Jones absence from OTAs in a year where a new coaching staff is trying to install a new defensive scheme.  But that aside, I continue to be astounded that Jones is still a part of this team at all.

I understand that cutting Jones post-June 1 would entail \href{https://overthecap.com/salary-cap/miami-dolphins/}{eating \$17 million dollars in dead cap money}.  But the Dolphins have plenty of cap space.

How can you keep a guy who basically quit on your team by refusing to enter a game last year?  I don't care what issues Jones had with defensive coordinator Matt Burke or head coach Adam Gase.  As Flores, himself, correctly points out above, none of your 53 ``leaders'' should ever put pride ahead of the team, especially if you are a serving as veteran example.

I understand that football players aren't all going to be angels.  But right now the focus has to be first and foremost on developing young players and establishing a positive team culture.

Similar to \href{https://www.phinmaniacs.com/news/running-back-mark-walton-gets-second-chance-with-miami-dolphins}{the signing troubled running back Mark Walton}, I can't imagine what \href{https://www.phinmaniacs.com/news/points-of-view-from-a-dolphins-perspective}{a poor example} such ``veteran leadership'' will set for the youth on a rebuilding Dolphins team.

Every time Flores speaks to the team about ``putting the team first'', young eyes in that room will involuntarily turn to look at Jones (when he finally shows up).  I can't imagine the damage that will cause.

\item Salguero \href{https://www.miamiherald.com/sports/spt-columns-blogs/armando-salguero/article230468024.html}{comments on the release of 2017 fifth round pick Isaac Asiata}:  

\begin{quote}
``[T]he frustrating thing about this pick is that Asiata was drafted out of Utah with the 164th selection. And with the 190th selection the Los Angeles Chargers selected Sam Tevi, who was on the same offensive line as Asiata at Utah.

``And Tevi was a backup as a rookie. And last year he started 15 games and won the starting right tackle job for the Chargers ahead of Joe Barksdale, who opened the door for Tevi when he got injured and then was cut when he got healthy because Tevi won the job outright.

``Somebody in the Dolphins’ personnel department studied the same Utah offensive line and picked the wrong guy.''
\end{quote}

In fairness, the Dolphins were still 2 years away from losing Ja'Wuan James to free agency.  They had just taken Laremy Tunsil in the first round in 2016 and they were just moving him to tackle.  They needed a guard.

Nevertheless, Salguero's point is excellent and well-taken.  Talent is talent and it isn't like the Dolphins couldn't have used a good developmental prospect that could serve as a back up in the meantime.

The situation highlights a fundamental problem.  The general manager can set the general direction of the franchise and help make some final decisions on the players.  But Chris Grier, even if he's the right guy for the job, can't be everywhere.  It takes a village to put together a good draft and if the scouts can't recognize talent when they see it, it results in an organizational failure.  The general manager is just the face of it.

This is an issue when you consider the fact that the Dolphins chose to stick with Grier rather than embrace big changes by hiring from the outside.  All the little indians underneath Grier stayed the same.  Here's hoping they improve their performance or it won't matter who the chief is.


\item Adam Beasley at the Miami Herald \href{https://www.miamiherald.com/sports/nfl/miami-dolphins/article230302419.html}{thinks he understands} the Dolphins strategy in building the offensive line. 

\begin{quote}
``[Michael] Deiter checks all those boxes.

``Wisconsin’s do-everything offensive lineman started a ridiculous 54 straight games in college.

``And the Badgers are no finesse team.

``Wisconsin averaged 43.9 rushing attempts in 2018, compared to just 23.4 passing attempts.

``So while there were more talented options available to the Dolphins before taking Deiter in the third round two weeks back, there might not have been a better option.

``We’re going to be a tough, physical team,' [Dolphins head coach] Brian Flores said during rookie mini-camp.

``The Dolphins drafted like it.''
\end{quote}

I think Beasley is right.  But I have a better explanation.

The Dolphins drafted Christian Wilkins, a defensive tackle, and Dieter, a guard, with their first two picks.  They are planning on being strong up the middle and building outward.  And that's a sound strategy.  Nothing disrupts the opposing offense more than pressure up the middle.  Offensive guard is being recognized as an increasingly important position in the modern NFL as teams struggle to keep the pocket clean in front of the quarterback so that he can step up and throw.

\item Omar Kelly at the Sun-Sentinel \href{points out that the Dolphins are deficient on the ends}{https://www.sun-sentinel.com/sports/miami-dolphins/fl-sp-dolphins-kelly-column-20190512-ud6r7pf2rjge5n54snql6or6z4-story.html} in the new defensive scheme.

\begin{quote}
  ``I get it. Rome wasn’t built in a day. And neither were the Patriots. But this is simple roster management issue that could be fixed with a veteran addition, similar to how Jordan Mills’ signing potentially patches up the vacant right tackle spot.''
\end{quote}

Kelly is primarily concerned with the run defense but the problem will crop up when it comes to the pass rush as well.  The Dolphins are deficient at this position, the outside linebacker when they are in the base 3-4 and the defensive end in the nickel.  This position sets the edge against the run and pushes the passer.

There's no doubt about it.  The Dolphins have a huge roster hole at this spot.  But I'd love to know who Kelly thinks the team should sign as a ``simple roster addition''.  The truth is most teams are deficient in this area.  Anyone who is any good was taken off the market a long time ago.  Even if the Dolphins were willing to part with 2020 draft picks, most of those players aren't available via trade.

The Dolphins have evidently made the decision to let this position slide in a rebuilding year.  I don't like it, Kelly doesn't like it and you probably don't like it, either.  But I think its something everyone is going to have to live with because it's not a problem that will be solved until the next offseason now.

\item Kelly and Safid Deen \href{https://www.sun-sentinel.com/dddc3139-323a-477a-84b2-550540d9d5e2-132.html}{discuss the development of the Dolphins defense} in this video (about the 2:35 mark).

There isn't much to say about it this early in the process but it is worth noting that installing a new defense and a new offense is going to be a particular challenge for this group of coaches.  The reason is simple - most of them haven't done it before.  Indeed, most of them have never even seen it done.

Many of the Dolphins coaches, both offensive and defensive, have spent their entire careers in New England, where head coach Bill Belichick is an institution and where the organization has been a smoothly running machine for a very long time.  Belichick probably doesn't even remember how he went about installing the Patriot defense from scratch, let alone his assistants.

The problem highlights one of the challenges that comes with being a Belichick disciple.  Many of these guys have never been anywhere else.

For instance, they've never developed the contacts in the league which can be necessary to succeed as coaches, particularly head coaches in Brian Flores's case.  So what happens when you have to hire assistants?  You look for former Patriots because you've never worked anywhere else and you don't really know anyone else.

Having veteran coaches like Jim Caldwell can help.  But that will only take you so far.  Flores and his staff have their work cut out for them in this respect.

\item Mike Florio at profootballtalk.com \href{https://profootballtalk.nbcsports.com/2019/05/17/rumors-fly-of-the-jets-pursuing-peyton-manning/}{explains what the Jets might have in mind for their general manager position}:

\begin{quote}
``Jets CEO and chairman Christopher Johnson wants a 'great strategic thinker' to run the football operation. He needs someone who can work with coach Adam Gase. And at the intersection possibly resides one and only one name.

Peyton Manning.
\end{quote}

This sounds to me more like the media connecting dots than a realistic possibility.  But I've been surprised before.

I love Peyton Manning but a general manager?  I'm not a big fan of having people without a background in personnel in that role, let alone someone with no front office background at all. It almost never works out.  The latest example is in San Francisco where rumor has it that John Lynch, who also had no front office experience, and head coach Kyle Shanahan are rumored to be on the outs.

I don’t like the direction the Jets are taking.

\item Florio also \href{https://profootballtalk.nbcsports.com/2019/05/17/prosecution-appeals-suppression-order-in-kraft-case/}{explains why prosecutors are appealing a ruling in the case against Robert Kraft}, who is accused of solicitation in Florida.

\begin{quote}
``Multiple judges have ruled that the “sneak and peek” video surveillance violated the law by undertaking no effort to minimize the intrusion on the privacy of innocent persons who were simply getting massages. If the appellate courts don’t overturn these rulings, there will be little or no evidence against Kraft — unless prosecutors can persuade the alleged providers of prostitution to “flip” on their alleged customers.''
\end{quote}

I have no interest in this case except that it bothers me when someone tries to legally get off the hook based upon technicalities.  I know he did it.  You know he did it.  The lawyers know he did it.

Kraft is the owner of a franchise where players are constantly told to be accountable for their actions.  Is this accountability?  It might be the reality of the world we live it.  But I call it hypocrisy.

In any case, the situation \href{https://profootballtalk.nbcsports.com/2019/05/16/if-robert-kraft-is-exonerated-what-happens-next/}{puts the league in a bind}.  They  haven't hesitated to suspend players who are obviously guilty but who have not been legally convicted, often because they paid off the victim.  Pittsburgh quarterback Ben Roethlisberger's \href{https://en.wikipedia.org/wiki/Ben_Roethlisberger#Sexual_assault_allegations}{2010 suspension after sexual assault allegations} is a good example.

I think its fair to say that although the league has some morally upright fans who strongly disapprove, solicitation isn't really considered to be a big deal to most in modern American society.  It certainly doesn't rise to the level of sexual assault or similar offenses.  But in terms of obvious guilt or innocence beyond the legal ramifications, there are players who are going to be watching this situation closely to see if Kraft is held to the same standard.
\end{itemize}
\end{document}