% Created 2019-06-24 Mon 10:31
% Intended LaTeX compiler: pdflatex
\documentclass[11pt]{article}
\usepackage[utf8]{inputenc}
\usepackage[T1]{fontenc}
\usepackage{graphicx}
\usepackage{grffile}
\usepackage{longtable}
\usepackage{wrapfig}
\usepackage{rotating}
\usepackage[normalem]{ulem}
\usepackage{amsmath}
\usepackage{textcomp}
\usepackage{amssymb}
\usepackage{capt-of}
\usepackage{hyperref}
\author{Tom Shannon}
\date{\today}
\title{}
\hypersetup{
 pdfauthor={Tom Shannon},
 pdftitle={},
 pdfkeywords={},
 pdfsubject={},
 pdfcreator={Emacs 26.2 (Org mode 9.1.9)}, 
 pdflang={English}}
\begin{document}

\tableofcontents

\section{\textbf{Ticklers}}
\label{sec:orga504bbf}
\subsection{Consider re-uping mantenance agreement for IDL \href{//\%3c0625ed0fbf874548891800b852e2a7f5@MLBXCH15.cs.myharris.net\%3E}{Quote No. QTO-042209}}
\label{sec:orgaf9b9aa}

\subsection{{\bfseries\sffamily DONE} FU: Don Bers \href{//\%3cF72116E9-F896-444D-A834-02403F73F1FD@rush.edu\%3E}{Could we set up a call?}}
\label{sec:org11734d7}
\section{\textbf{Meetings}}
\label{sec:org5775d57}
\subsection{Phone call with Sabine Van Dijk}
\label{sec:org4d82561}
\textit{<2018-01-15 Mon>}
\subsubsection{Pre-call questions/notes}
\label{sec:orgdefd450}
\begin{enumerate}
\item I assume these are cardiac myocytes?  How well was AKAP over-expressed?
\label{sec:orga18e98b}
\begin{enumerate}
\item Yes.  Don't really know - nothing to compare it to.
\label{sec:org05931ae}
\end{enumerate}
\item What worries me is that all of the taus in the grant are lower except the H89.  Its as if the forskolin wasn't working.
\label{sec:org44d7704}
\item Do the cells respond physiologically to the forskolin?
\label{sec:org5321af8}
\begin{enumerate}
\item She will check this.
\label{sec:orge6d7a7a}
\end{enumerate}
\item What is your data acquisition rate? Can you send me a data set?
\label{sec:org9bd29af}
\begin{enumerate}
\item She talked about sending me one but I didn't re-enforce it so it may not happen.
\label{sec:orgccd2a85}
\end{enumerate}
\item Can we try photobleaching more of the cell?
\label{sec:org6d0c0f4}
\begin{enumerate}
\item Will try
\label{sec:org40cde91}
\end{enumerate}
\item What happens with PLB over-expression?  Can we do a double expression experiment?
\label{sec:org0c1d3b0}
\begin{enumerate}
\item Maybe for the future
\label{sec:orgb5874f8}
\end{enumerate}
\item myotubes are bigger\ldots{}  Should we try those?
\label{sec:org690a32c}
\begin{enumerate}
\item Maybe later
\label{sec:orge4a4449}
\end{enumerate}
\item Should we talk again?
\label{sec:orge58410a}
\begin{enumerate}
\item Yes.  Circumstances dictated 3 weeks to 1 month
\label{sec:orgdb159d7}
\end{enumerate}
\item Do you have other projects?
\label{sec:org52f5015}
\begin{enumerate}
\item She does.  They are AKAP related and they aren't going well either but all of her eggs aren't in this basket
\label{sec:orgaf44af1}
\end{enumerate}
\end{enumerate}
\subsection{Phone call with Don Bers (530) 752-6517‬ 2018-05-22}
\label{sec:org11bde74}
\subsubsection{Struggling}
\label{sec:orgc46ff1b}
\begin{enumerate}
\item We weren't getting adequate temporal resolution but now measuring time to 50\% decline for every transient so we think that issue is solved.
\label{sec:org01c4a4c}
\item Cells tend to die upon rapid switch (Did Lipsius do rapid switch?)
\label{sec:orgb71a2e6}
\item Cells tend to stop contracting.
\label{sec:orgfcdfd8a}
\item Not at all sure Dan has the most efiicent set up - sometimes misses cells and we get no response.
\label{sec:orgf10d46f}
\end{enumerate}
\subsubsection{Wanted to talk before Friday so that a plan would be in place}
\label{sec:org607aa59}
\subsubsection{Would suggest that Dan visit California for at least 2 weeks in July}
\label{sec:org2b58d46}
\begin{enumerate}
\item Would like for him to see how you guys set this experiment up.
\label{sec:orgcdbdd32}
\item Might be helpful for you to see what's going wrong
\label{sec:org9470a2c}
\item Forces seperation from other lab and allows Dan to concentrate on the experiments.
\label{sec:org6a8967a}
\end{enumerate}
\subsubsection{Don agreed with this plan.  Suggested I come out for a couple days toward the end of the visit.}
\label{sec:org1a9e7ca}
\subsection{AKAP Meeting}
\label{sec:org8ba118e}
\subsubsection{Ken and Sabine doing ISO and epinephrine kinetics on AKAP KO.}
\label{sec:org5d7226f}
\subsubsection{Interfering peptide with FRAP}
\label{sec:org9d81661}
\subsubsection{Tom - FRAP experiments}
\label{sec:orgf0ce34f}
\subsubsection{Sabina is seeing striations with AKAP expression - I never did.}
\label{sec:orgb7aac4a}
\subsubsection{She will try decreasing exprssion time.}
\label{sec:org20aad9b}
\subsubsection{Sending Dan to Bers Lab in July to work with Ken.  I will join the last couple days}
\label{sec:orgb4634cd}
\subsection{CRC Laboratory Representative Meeting; Cohn Auditorium \textit{<2019-04-05 Fri 11:00>--<2019-04-05 Fri 12:00>}}
\label{sec:org993d536}
\section{\textbf{Notes}}
\label{sec:org058c930}
\subsection{You were having trouble with 2019-06=19-12.  You were just about to reanalyze with new parameters for 25 micron band.  Try these.}
\label{sec:orgb46c55f}
\subsection{Ideas for Dan}
\label{sec:orgd931f96}
\subsubsection{TG to inhibit phosphorylation of some of the PLB}
\label{sec:orgc4d4934}
\subsubsection{Partial Ca channel block}
\label{sec:org0fab6cb}
\subsubsection{Even slower stimulation rate}
\label{sec:org74c5a2a}
\subsection{The Biophysicist?}
\label{sec:org7255fce}
\href{//\%3cLYRIS-20262805-43074-2019.04.16-15.41.20--tom\_shannon\#rush.edu@lists.biophysics.org\%3E}{Announcing The Biophysicist: A New Journal from BPS}
\subsection{2019}
\label{sec:org6b2905d}
\subsubsection{963.64 us/line for this data}
\label{sec:orgec23fb9}
\subsubsection{2019-05}
\label{sec:orgb39e4d3}
\begin{enumerate}
\item 2019-05-31
\label{sec:orga8200bf}
\begin{enumerate}
\item 2019-05-31-4
\label{sec:orgdfd5a17}
\begin{enumerate}
\item \begin{center}
\includegraphics[width=.9\linewidth]{/Users/tshanno/Library/Mobile Documents/com~apple~CloudDocs/zFiled By Folder/Shannon Data/2019-05-31/Plot of Rhod2-LineScan_ISO_053119_4.png}
\end{center}
\label{sec:org64db814}
\item \begin{center}
\includegraphics[width=.9\linewidth]{/Users/tshanno/Library/Mobile Documents/com~apple~CloudDocs/zFiled By Folder/Shannon Data/2019-05-31/Plot of Rhod2-LineScan_ISO_053119_4B.png}
\end{center}
\label{sec:org165647f}
\item \begin{center}
\includegraphics[width=.9\linewidth]{/Users/tshanno/Library/Mobile Documents/com~apple~CloudDocs/zFiled By Folder/Shannon Data/2019-05-31/Plot of Rhod2-LineScan_ISO_053119_4C.png}
\end{center}
\label{sec:orga5aa629}
\item Supper noisy transients.  Even if there was something there I don't think I'd pick it up.
\label{sec:org85631c6}
\item Did not analyze B and C
\label{sec:orge971864}
\begin{enumerate}
\item \begin{center}
\includegraphics[width=.9\linewidth]{/Users/tshanno/Library/Mobile Documents/com~apple~CloudDocs/zFiled By Folder/Shannon Data/2019-05-31/Rhod2-LineScan_ISO_053119_4.pdf}
\end{center}
\label{sec:org3c93e00}
\item \begin{center}
\includegraphics[width=.9\linewidth]{/Users/tshanno/Library/Mobile Documents/com~apple~CloudDocs/zFiled By Folder/Shannon Data/2019-05-31/Rhod2-LineScan_ISO_053119_4.png}
\end{center}
\label{sec:orgd81ee98}
\item \href{file:///Users/tshanno/Library/Mobile Documents/com\~apple\~CloudDocs/zFiled By Folder/Shannon Data/2019-05-31/Rhod2-LineScan\_ISO\_053119\_4.pzf}{Rhod2-LineScan\(_{\text{ISO}}\)\(_{\text{053119}}\)\(_{\text{4.pzf}}\)}
\label{sec:org7be2ddf}
\end{enumerate}
\end{enumerate}
\item 2019-05-31-6
\label{sec:orgbf2fc2a}
\begin{enumerate}
\item \begin{center}
\includegraphics[width=.9\linewidth]{/Users/tshanno/Library/Mobile Documents/com~apple~CloudDocs/zFiled By Folder/Shannon Data/2019-05-31/Plot of Rhod2-LineScan_ISO_053119_6.png}
\end{center}
\label{sec:org9a2147a}
\item \begin{center}
\includegraphics[width=.9\linewidth]{/Users/tshanno/Library/Mobile Documents/com~apple~CloudDocs/zFiled By Folder/Shannon Data/2019-05-31/Plot of Rhod2-LineScan_ISO_053119_6B.png}
\end{center}
\label{sec:orgc4298ee}
\item \begin{center}
\includegraphics[width=.9\linewidth]{/Users/tshanno/Library/Mobile Documents/com~apple~CloudDocs/zFiled By Folder/Shannon Data/2019-05-31/Rhod2-LineScan_ISO_053119_6.pdf}
\end{center}
\label{sec:org14ee376}
\item \begin{center}
\includegraphics[width=.9\linewidth]{/Users/tshanno/Library/Mobile Documents/com~apple~CloudDocs/zFiled By Folder/Shannon Data/2019-05-31/Rhod2-LineScan_ISO_053119_6.png}
\end{center}
\label{sec:org8e5c635}
\item \href{file:///Users/tshanno/Library/Mobile Documents/com\~apple\~CloudDocs/zFiled By Folder/Shannon Data/2019-05-31/Rhod2-LineScan\_ISO\_053119\_6.pzf}{Rhod2-LineScan\(_{\text{ISO}}\)\(_{\text{053119}}\)\(_{\text{6.pzf}}\)}
\label{sec:org8516b6c}
\end{enumerate}
\item 2019-05-31-7
\label{sec:org15aebb7}
\begin{enumerate}
\item \begin{center}
\includegraphics[width=.9\linewidth]{/Users/tshanno/Library/Mobile Documents/com~apple~CloudDocs/zFiled By Folder/Shannon Data/2019-05-31/Plot of Rhod2-LineScan_ISO_053119_7.png}
\end{center}
\label{sec:org852ca96}
\item This one didn't have transients that appeared to shorten that much.
\label{sec:orgddb0213}
\item \begin{center}
\includegraphics[width=.9\linewidth]{/Users/tshanno/Library/Mobile Documents/com~apple~CloudDocs/zFiled By Folder/Shannon Data/2019-05-31/2019-05-31-7.pdf}
\end{center}
\label{sec:orgdd47f4e}
\item This analysis didn't go well.
\label{sec:org371da42}
\begin{enumerate}
\item \begin{center}
\includegraphics[width=.9\linewidth]{/Users/tshanno/Library/Mobile Documents/com~apple~CloudDocs/zFiled By Folder/Shannon Data/2019-05-31/Rhod2-LineScan_ISO_053119_7.pdf}
\end{center}
\label{sec:org3c2925a}
\item \begin{center}
\includegraphics[width=.9\linewidth]{/Users/tshanno/Library/Mobile Documents/com~apple~CloudDocs/zFiled By Folder/Shannon Data/2019-05-31/Rhod2-LineScan_ISO_053119_7.png}
\end{center}
\label{sec:org758c8b0}
\item \href{file:///Users/tshanno/Library/Mobile Documents/com\~apple\~CloudDocs/zFiled By Folder/Shannon Data/2019-05-31/Rhod2-LineScan\_ISO\_053119\_7.pzf}{Rhod2-LineScan\(_{\text{ISO}}\)\(_{\text{053119}}\)\(_{\text{7.pzf}}\)}
\label{sec:org1be8715}
\item The x-axis appers to be messed up so I was guessing as to how many lines in I should make the first transient.
\label{sec:orgac3e4d5}
\item Had to round the interval between beats to an integer which probably introduced a small time error that accumulated over many beats.
\label{sec:org248f3d2}
\item It also appears that the stimulation rate wasn't exactly 0.5 Hz.
\label{sec:orgc309272}
\end{enumerate}
\end{enumerate}
\end{enumerate}
\end{enumerate}
\subsubsection{2019-06}
\label{sec:orgff526fb}
\begin{enumerate}
\item 2019-06-04
\label{sec:org5e22c6b}
\begin{enumerate}
\item 2019-06-04-1
\label{sec:org3204071}
\begin{enumerate}
\item Initial Analysis
\label{sec:orgc60d0ba}
\begin{center}
\includegraphics[width=.9\linewidth]{/Users/tshanno/Library/Mobile Documents/com~apple~CloudDocs/zFiled By Folder/Shannon Data/2019-06-04/Plot of Rhod2-LineScan_ISO_060419_1.png}
\end{center}
\begin{center}
\includegraphics[width=.9\linewidth]{/Users/tshanno/Library/Mobile Documents/com~apple~CloudDocs/zFiled By Folder/Shannon Data/2019-06-04/Rhod2-LineScan_ISO_060419_1.pdf}
\end{center}
\begin{center}
\includegraphics[width=.9\linewidth]{/Users/tshanno/Library/Mobile Documents/com~apple~CloudDocs/zFiled By Folder/Shannon Data/2019-06-04/Rhod2-LineScan_ISO_060419_1.png}
\end{center}
\begin{enumerate}
\item \href{file:///Users/tshanno/Library/Mobile Documents/com\~apple\~CloudDocs/zFiled By Folder/Shannon Data/2019-06-04/Rhod2-LineScan\_ISO\_060419\_1.pzf}{Rhod2-LineScan\(_{\text{ISO}}\)\(_{\text{060419}}\)\(_{\text{1.pzf}}\)}
\label{sec:org62dd5a0}
\end{enumerate}
\item Reanalysis 2019-06-19
\label{sec:orga55955f}
\begin{enumerate}
\item 2019-06-04-1 and -2 This data's no good.  It looks like there might be a gradient in the transients but the peaks are almost random as a function of space.
\label{sec:org698e9f2}
\begin{center}
\includegraphics[width=.9\linewidth]{/Users/tshanno/Library/Mobile Documents/com~apple~CloudDocs/zFiled By Folder/Shannon Data/Reanalysis 2019-06-19/2019-06-04-1/Analysis 25 micron moving average/Rhod2-LineScan_ISO_060419_1.lsm_TimeVsNormalizedTransientPeak.png}
\end{center}
\end{enumerate}
\end{enumerate}
\item 2019-06-04-02
\label{sec:org2263b40}
\begin{enumerate}
\item Initial Analysis
\label{sec:org6fb2274}
\begin{center}
\includegraphics[width=.9\linewidth]{/Users/tshanno/Library/Mobile Documents/com~apple~CloudDocs/zFiled By Folder/Shannon Data/2019-06-04/Plot of Rhod2-LineScan_ISO_060419_2.png}
\end{center}
\begin{center}
\includegraphics[width=.9\linewidth]{/Users/tshanno/Library/Mobile Documents/com~apple~CloudDocs/zFiled By Folder/Shannon Data/2019-06-04/Rhod2-LineScan_ISO_060419_2.pdf}
\end{center}
\begin{center}
\includegraphics[width=.9\linewidth]{/Users/tshanno/Library/Mobile Documents/com~apple~CloudDocs/zFiled By Folder/Shannon Data/2019-06-04/Rhod2-LineScan_ISO_060419_2.png}
\end{center}
\begin{enumerate}
\item \href{file:///Users/tshanno/Library/Mobile Documents/com\~apple\~CloudDocs/zFiled By Folder/Shannon Data/2019-06-04/Rhod2-LineScan\_ISO\_060419\_2.pzf}{Rhod2-LineScan\(_{\text{ISO}}\)\(_{\text{060419}}\)\(_{\text{2.pzf}}\)}
\label{sec:org4b9c29e}
\end{enumerate}
\item Reanalysis 2019-06-19
\label{sec:org8f9983b}
\begin{enumerate}
\item 2019-06-04-1 and -2 This data's no good.  It looks like there might be a gradient in the transients but the peaks are almost random as a function of space.
\label{sec:orgcadd5a4}
\begin{center}
\includegraphics[width=.9\linewidth]{/Users/tshanno/Library/Mobile Documents/com~apple~CloudDocs/zFiled By Folder/Shannon Data/Reanalysis 2019-06-19/2019-06-04-2/Analysis 25 micron moving average/Rhod2-LineScan_ISO_060419_2.lsm_TimeVsNormalizedTransientPeak.png}
\end{center}
\end{enumerate}
\end{enumerate}
\end{enumerate}
\item 2019-06-06
\label{sec:org1bad740}
\begin{enumerate}
\item 2019-06-06-5
\label{sec:org95c323d}
\begin{enumerate}
\item Initial Analysis
\label{sec:org879650d}
\begin{center}
\includegraphics[width=.9\linewidth]{/Users/tshanno/Library/Mobile Documents/com~apple~CloudDocs/zFiled By Folder/Shannon Data/2019-06-06/Plot of Rhod2-LineScan_ISO_060619_5.png}
\end{center}
\begin{enumerate}
\item Seems to get shorter
\label{sec:org8819ca7}
\begin{enumerate}
\item \begin{center}
\includegraphics[width=.9\linewidth]{/Users/tshanno/Library/Mobile Documents/com~apple~CloudDocs/zFiled By Folder/Shannon Data/2019-06-06/Rhod2-LineScan_ISO_060619_5.pdf}
\end{center}
\label{sec:org044c0e6}
\item \begin{center}
\includegraphics[width=.9\linewidth]{/Users/tshanno/Library/Mobile Documents/com~apple~CloudDocs/zFiled By Folder/Shannon Data/2019-06-06/Rhod2-LineScan_ISO_060619_5.png}
\end{center}
\label{sec:org563cbf3}
\item \href{file:///Users/tshanno/Library/Mobile Documents/com\~apple\~CloudDocs/zFiled By Folder/Shannon Data/2019-06-06/Rhod2-LineScan\_ISO\_060619\_5.pzf}{Rhod2-LineScan\(_{\text{ISO}}\)\(_{\text{060619}}\)\(_{\text{5.pzf}}\)}
\label{sec:orga1e3188}
\item Hmmmm\ldots{}  BAnd 2 may have shortened a bit but no response from Band 1 at all.  Odd cell.
\label{sec:orge2079f0}
\end{enumerate}
\end{enumerate}
\item Reanalysis 2019-06-19
\label{sec:org892f602}
\begin{enumerate}
\item pretty noisy.  The peak transient data migh tbe useful.
\label{sec:orgf5c2d19}
\begin{center}
\includegraphics[width=.9\linewidth]{/Users/tshanno/Library/Mobile Documents/com~apple~CloudDocs/zFiled By Folder/Shannon Data/Reanalysis 2019-06-19/2019-06-06-5/Analysis 25 micron moving average/Rhod2-LineScan_ISO_060619_5.lsm_TimeVsNormalizedTransientPeak.png}
\end{center}
\end{enumerate}
\end{enumerate}
\end{enumerate}
\item 2019-06-07
\label{sec:org4dc39f4}
\begin{enumerate}
\item Initial Analysis
\label{sec:org57a14bd}
\begin{enumerate}
\item The analysis of these cells indicated a possible affect on the decline to 50\%.  I'm not sure , though.  The  linear nature of the decline makes me think that as the transients gradullay become more off center the time to 90\% gets shorter just because the transient never gets to 90\% of the pak i.e. the "time to 90\%" is actually the time to the end of the segment.  Still, its always band 3 that seems to be hitting this limit.  So the decline must still be slower\ldots{}
\label{sec:org21e6555}
\item \textbf{I think we need some controls}
\label{sec:org1bd4fc6}
\end{enumerate}
\item 2019-06-07-2
\label{sec:org2132be7}
\begin{enumerate}
\item Initial Analysis
\label{sec:orgee5613b}
\begin{center}
\includegraphics[width=.9\linewidth]{/Users/tshanno/Library/Mobile Documents/com~apple~CloudDocs/zFiled By Folder/Shannon Data/2019-06-07/Plot of Rhod2-LineScan_ISO_060719_2.png}
\end{center}
\begin{enumerate}
\item Defnitely shorter
\label{sec:orgc08e346}
\item \textit{[2019-06-10 Mon] } Analysis indicated no difference in time to half decline but there might be a difference in the time to 90\% decline.  I may have to smooth this data a bit.
\label{sec:org4ae7f59}
\begin{center}
\includegraphics[width=.9\linewidth]{/Users/tshanno/Library/Mobile Documents/com~apple~CloudDocs/zFiled By Folder/Shannon Data/2019-06-07/Rhod2-LineScan_ISO_060719_2.pdf}
\end{center}
\begin{center}
\includegraphics[width=.9\linewidth]{/Users/tshanno/Library/Mobile Documents/com~apple~CloudDocs/zFiled By Folder/Shannon Data/2019-06-07/Rhod2-LineScan_ISO_060719_2.png}
\end{center}
\begin{enumerate}
\item \href{file:///Users/tshanno/Library/Mobile Documents/com\~apple\~CloudDocs/zFiled By Folder/Shannon Data/2019-06-07/Rhod2-LineScan\_ISO\_060719\_2.pzf}{Rhod2-LineScan\(_{\text{ISO}}\)\(_{\text{060719}}\)\(_{\text{2.pzf}}\)}
\label{sec:org0740d80}
\end{enumerate}
\item \textit{[2019-06-11 Tue] } Reanalysis
\label{sec:org8994dbd}
\begin{enumerate}
\item Reanalysis 1 was simply playing with the times to center the transients
\label{sec:org7266940}
\begin{enumerate}
\item \href{file:///Users/tshanno/Library/Mobile Documents/com\~apple\~CloudDocs/zFiled By Folder/Shannon Data/2019-06-07/Rhod2-LineScan\_ISO\_060719\_2R1.pzf}{Rhod2-LineScan\(_{\text{ISO}}\)\(_{\text{060719}}\)\(_{\text{2R1.pzf}}\)}
\label{sec:org4a4b3f3}
\item \begin{center}
\includegraphics[width=.9\linewidth]{/Users/tshanno/Library/Mobile Documents/com~apple~CloudDocs/zFiled By Folder/Shannon Data/2019-06-07/Rhod2-LineScan_ISO_060712_2R1.png}
\end{center}
\label{sec:org16452d6}
\end{enumerate}
\item Reanalysis 2 was reanalysis 1 with a 9 point box car smooth
\label{sec:org4c75135}
\begin{enumerate}
\item \begin{center}
\includegraphics[width=.9\linewidth]{/Users/tshanno/Library/Mobile Documents/com~apple~CloudDocs/zFiled By Folder/Shannon Data/2019-06-07/Rhod2-LineScan_ISO_060719_2R2.png}
\end{center}
\label{sec:org4c0d510}
\item \href{file:///Users/tshanno/Library/Mobile Documents/com\~apple\~CloudDocs/zFiled By Folder/Shannon Data/2019-06-07/Rhod2-LineScan\_ISO\_060719\_2R1.pzf}{Rhod2-LineScan\(_{\text{ISO}}\)\(_{\text{060719}}\)\(_{\text{2R1.pzf}}\)}
\label{sec:org5f2ce97}
\end{enumerate}
\end{enumerate}
\end{enumerate}
\item Reanalysis 2019-06-18
\label{sec:orgd58a6a8}
\begin{center}
\includegraphics[width=.9\linewidth]{/Users/tshanno/Library/Mobile Documents/com~apple~CloudDocs/zFiled By Folder/Shannon Data/Reanalysis 2019-06-18/2019-06-07-2/Analysis 25 micron moving average 2019-06-18/Rhod2-LineScan_ISO_060719_2.lsm_TimeVsNormalizedTimeTo90.png}
\end{center}

\begin{center}
\includegraphics[width=.9\linewidth]{/Users/tshanno/Library/Mobile Documents/com~apple~CloudDocs/zFiled By Folder/Shannon Data/Reanalysis 2019-06-18/2019-06-07-2/Analysis 25 micron moving average 2019-06-18/Rhod2-LineScan_ISO_060719_2.lsm_TimeVsNormalizedTimeToHalf.png}
\end{center}

\begin{center}
\includegraphics[width=.9\linewidth]{/Users/tshanno/Library/Mobile Documents/com~apple~CloudDocs/zFiled By Folder/Shannon Data/Reanalysis 2019-06-18/2019-06-07-2/Analysis 25 micron moving average 2019-06-18/Rhod2-LineScan_ISO_060719_2.lsm_TimeVsNormalizedTransientPeak.png}
\end{center}
\begin{enumerate}
\item There might be a gradient in the tt90 data but I wouldn't bet that it will amount to anything.   Nothing in the tt50 data.
\label{sec:orgac45daa}

The transient peak data shows a definite gradient.  The 25 um band data looks better.
\end{enumerate}
\end{enumerate}
\item 2019-06-07-3
\label{sec:org9e20817}
\begin{enumerate}
\item Initial Analysis
\label{sec:org70b6b85}
\begin{center}
\includegraphics[width=.9\linewidth]{/Users/tshanno/Library/Mobile Documents/com~apple~CloudDocs/zFiled By Folder/Shannon Data/2019-06-07/Plot of Rhod2-LineScan_ISO_060719_3.png}
\end{center}
\begin{enumerate}
\item Reaction to ISO not as apparent as 2
\label{sec:orga04f262}
\item Couldn't quite get all of the cell into the analysis.  It looked like there was still some fluorescence left at the very end of the spacial profile.
\label{sec:org76e2498}
\begin{enumerate}
\item \begin{center}
\includegraphics[width=.9\linewidth]{/Users/tshanno/Library/Mobile Documents/com~apple~CloudDocs/zFiled By Folder/Shannon Data/2019-06-07/Rhod2-LineScan_ISO_060719_3.pdf}
\end{center}
\label{sec:org018c27e}
\item \begin{center}
\includegraphics[width=.9\linewidth]{/Users/tshanno/Library/Mobile Documents/com~apple~CloudDocs/zFiled By Folder/Shannon Data/2019-06-07/Rhod2-LineScan_ISO_060719_3.png}
\end{center}
\label{sec:orgf4545f4}
\item \href{file:///Users/tshanno/Library/Mobile Documents/com\~apple\~CloudDocs/zFiled By Folder/Shannon Data/2019-06-07/Rhod2-LineScan\_ISO\_060719\_3.pzf}{Rhod2-LineScan\(_{\text{ISO}}\)\(_{\text{060719}}\)\(_{\text{3.pzf}}\)}
\label{sec:org27d3575}
\item Again, it looks like the difference is in the time to 90\% decline.  I know that some of the transients at the end of this must be messed up because I was only getting half of the transient to analyze.  Still the difference looks like its there in the early data when I know the analysis is good.
\label{sec:org0869e48}
\end{enumerate}
\item Reanalyzed this one twice as with 2019-06-07-2
\label{sec:org07dd319}
\begin{enumerate}
\item \begin{center}
\includegraphics[width=.9\linewidth]{/Users/tshanno/Library/Mobile Documents/com~apple~CloudDocs/zFiled By Folder/Shannon Data/2019-06-07/Rhod2-LineScan_ISO_060719_3R1.png}
\end{center}
\label{sec:org4946498}
\item \href{file:///Users/tshanno/Library/Mobile Documents/com\~apple\~CloudDocs/zFiled By Folder/Shannon Data/2019-06-07/Rhod2-LineScan\_ISO\_060719\_3R1.pzf}{Rhod2-LineScan\(_{\text{ISO}}\)\(_{\text{060719}}\)\(_{\text{3R1.pzf}}\)}
\label{sec:org41a5947}
\item \href{file:///Users/tshanno/Library/Mobile Documents/com\~apple\~CloudDocs/zFiled By Folder/Shannon Data/2019-06-07/Rhod2-LineScan\_ISO\_060719\_3R2.pzf}{Rhod2-LineScan\(_{\text{ISO}}\)\(_{\text{060719}}\)\(_{\text{3R2.pzf}}\)}
\label{sec:org126e9a4}
\item \begin{center}
\includegraphics[width=.9\linewidth]{/Users/tshanno/Library/Mobile Documents/com~apple~CloudDocs/zFiled By Folder/Shannon Data/2019-06-07/Rhod2-LineScan_ISO_060719_3R2.png}
\end{center}
\label{sec:orgf6ca89c}
\end{enumerate}
\end{enumerate}
\item Reanalysis 2019-06-18
\label{sec:org6002b55}
\begin{center}
\includegraphics[width=.9\linewidth]{/Users/tshanno/Library/Mobile Documents/com~apple~CloudDocs/zFiled By Folder/Shannon Data/Reanalysis 2019-06-18/2019-06-07-3/Spatial profile.png}
\end{center}
\begin{enumerate}
\item Note that the profile for this cell went all the way to the edge.  Used 500.
\label{sec:org72f01db}
\item \begin{center}
\includegraphics[width=.9\linewidth]{/Users/tshanno/Library/Mobile Documents/com~apple~CloudDocs/zFiled By Folder/Shannon Data/Reanalysis 2019-06-18/2019-06-07-3/Reanalysis 10 micron moving average 2019-06-18/Rhod2-LineScan_ISO_060719_3.lsm_TimeVsTransientPeak.png}
\end{center}
\label{sec:orgfcdc4bc}
\item \begin{center}
\includegraphics[width=.9\linewidth]{/Users/tshanno/Library/Mobile Documents/com~apple~CloudDocs/zFiled By Folder/Shannon Data/Reanalysis 2019-06-18/2019-06-07-3/Reanalysis 10 micron moving average 2019-06-18/Rhod2-LineScan_ISO_060719_3.lsm_TimeVsNormalizedTransientPeak.png}
\end{center}
\label{sec:orgab5132d}
\item This data is noisy but there's a gradient there.  Unfortunately the peaks were unsteady and a bit up and down.  I don't know how usable this data will be.
\label{sec:org0373ec0}
\item \begin{center}
\includegraphics[width=.9\linewidth]{/Users/tshanno/Library/Mobile Documents/com~apple~CloudDocs/zFiled By Folder/Shannon Data/Reanalysis 2019-06-18/2019-06-07-3/Reanalysis 25 micron moving average 2019-06-18/Rhod2-LineScan_ISO_060719_3.lsm_TimeVsNormalizedTimeTo90.png}
\end{center}
\label{sec:orgba00414}
\item \begin{center}
\includegraphics[width=.9\linewidth]{/Users/tshanno/Library/Mobile Documents/com~apple~CloudDocs/zFiled By Folder/Shannon Data/Reanalysis 2019-06-18/2019-06-07-3/Reanalysis 25 micron moving average 2019-06-18/Rhod2-LineScan_ISO_060719_3.lsm_TimeVsNormalizedTimeToHalf.png}
\end{center}
\label{sec:org7ef513d}
\item Can probably use the tt90 data but not the tt50.
\label{sec:org069b4bc}
\item \begin{center}
\includegraphics[width=.9\linewidth]{/Users/tshanno/Library/Mobile Documents/com~apple~CloudDocs/zFiled By Folder/Shannon Data/Reanalysis 2019-06-18/2019-06-07-3/Reanalysis 25 micron moving average 2019-06-18/Rhod2-LineScan_ISO_060719_3.lsm_BandVsNormalizedTimeTo90.png}
\end{center}
\label{sec:org444e8dd}
\item ISO apparently hit the cell about halfway up.
\label{sec:org9997fab}
\end{enumerate}
\end{enumerate}
\item 2019-06-07-4
\label{sec:orgc38daa1}
\begin{enumerate}
\item Initial Analysis
\label{sec:org74faef3}
\begin{center}
\includegraphics[width=.9\linewidth]{/Users/tshanno/Library/Mobile Documents/com~apple~CloudDocs/zFiled By Folder/Shannon Data/2019-06-07/Plot of Rhod2-LineScan_ISO_060719_4.png}
\end{center}
\begin{enumerate}
\item Again reaction not as apparent as 2.
\label{sec:orgbf16688}
\item Upon analysis, I don't think anything is here.
\label{sec:org2641cec}
\begin{enumerate}
\item \begin{center}
\includegraphics[width=.9\linewidth]{/Users/tshanno/Library/Mobile Documents/com~apple~CloudDocs/zFiled By Folder/Shannon Data/2019-06-07/Rhod2-LineScan_ISO_060719_4.pdf}
\end{center}
\label{sec:org56521cc}
\item \begin{center}
\includegraphics[width=.9\linewidth]{/Users/tshanno/Library/Mobile Documents/com~apple~CloudDocs/zFiled By Folder/Shannon Data/2019-06-07/Rhod2-LineScan_ISO_060719_4.png}
\end{center}
\label{sec:orga76e4c0}
\item \href{file:///Users/tshanno/Library/Mobile Documents/com\~apple\~CloudDocs/zFiled By Folder/Shannon Data/2019-06-07/Rhod2-LineScan\_ISO\_060719\_4.pzf}{Rhod2-LineScan\(_{\text{ISO}}\)\(_{\text{060719}}\)\(_{\text{4.pzf}}\)}
\label{sec:org4311606}
\item Maybe a slower decline to 90\% in band 3 as with the other cells.  But the linear nature is suspicious.
\label{sec:orgba6e355}
\end{enumerate}
\end{enumerate}
\item Reanalysis 2019-06-18
\label{sec:org42665a2}
\begin{center}
\includegraphics[width=.9\linewidth]{/Users/tshanno/Library/Mobile Documents/com~apple~CloudDocs/zFiled By Folder/Shannon Data/Reanalysis 2019-06-18/2019-06-07-4/Reanalysis 25 micron moving average 2019-06-18/Rhod2-LineScan_ISO_060719_4.lsm_TimeVsNormalizedTimeTo90.png}
\end{center}
\begin{enumerate}
\item The gradient is now evident here.  There's nothing thee in the time to hald decline state.
\label{sec:org02eb725}
\begin{center}
\includegraphics[width=.9\linewidth]{/Users/tshanno/Library/Mobile Documents/com~apple~CloudDocs/zFiled By Folder/Shannon Data/Reanalysis 2019-06-18/2019-06-07-4/Reanalysis 10 micron moving average 2019-06-18/Rhod2-LineScan_ISO_060719_4.lsm_TimeVsTransientPeak.png}
\end{center}
 \begin{center}
\includegraphics[width=.9\linewidth]{/Users/tshanno/Library/Mobile Documents/com~apple~CloudDocs/zFiled By Folder/Shannon Data/Reanalysis 2019-06-18/2019-06-07-4/Reanalysis 10 micron moving average 2019-06-18/Rhod2-LineScan_ISO_060719_4.lsm_TimeVsNormalizedTransientPeak.png}
\end{center}
\item It appears that there is also a gradient in the peak transient.
\label{sec:org916480d}
\item Both of these seem to show no response high up in the cell near the top.  The ISO response doesn't seem to propagate all the way throughout.
\label{sec:org696f2fb}
\end{enumerate}
\end{enumerate}
\end{enumerate}
\item 2019-06-11
\label{sec:org39f180b}
\begin{itemize}
\item Note taken on \textit{[2019-06-16 Sun 09:13] } \\
These are all control cells where the ISO simply wasn't turned on.  So they are all just being stimulated at steady-state.  I just wanted to confirm that there aren't any relevant artifacts being added by the analysis.
\end{itemize}
\begin{enumerate}
\item \href{//\%3c9A0A9BA1-B4AA-4CB0-B52B-1D8F6E5A7BBF@rush.edu\%3E}{Re: More analysis}
\label{sec:org2d4f3a0}


I got 3-5 Control-Tyrode cells today, some better than others but it is clear that there was no increase in transients over the course of the run.  There was a shift with some either up or down a little at the start of the perfusion but no sign of any increase and in contrast for some the transient actually decreases toward the end which is not seen with the ISO cells.  So you can get the data tomorrow if possible I think that these will suit your purpose. 

Also, I'm doing a cell isolation again tomorrow and Thursday since I can not do one Friday ( Jiajie needs to use the confocal ) and will attempt to get more ISO perfused cells.
\item 209-06-11-05
\label{sec:org3586feb}
\begin{enumerate}
\item Initial Analysis
\label{sec:org5eafc99}
\begin{center}
\includegraphics[width=.9\linewidth]{/Users/tshanno/Library/Mobile Documents/com~apple~CloudDocs/zFiled By Folder/Shannon Data/2019-06-11/Plot of Rhod2-LineScan_CTRL_061119_5.png}
\end{center}
\begin{enumerate}
\item \href{file:///Users/tshanno/Library/Mobile Documents/com\~apple\~CloudDocs/zFiled By Folder/Shannon Data/2019-06-11/Rhod2-LineScan\_CTRL\_061119\_5.pzf}{Rhod2-LineScan\(_{\text{CTRL}}\)\(_{\text{061119}}\)\(_{\text{5.pzf}}\)}
\label{sec:org4d67d4a}
\begin{center}
\includegraphics[width=.9\linewidth]{/Users/tshanno/Library/Mobile Documents/com~apple~CloudDocs/zFiled By Folder/Shannon Data/2019-06-11/Analysis Rhod2-LineScan_CTRL_061119_5.png}
\end{center}
\end{enumerate}
\item Reanalysis
\label{sec:org1d5c8bc}
\begin{enumerate}
\item The reanalysis of this data using the moving average still showed no aritfacts due to analysis.
\label{sec:org0e38a27}
\end{enumerate}
\end{enumerate}
\item 2019-06-11-06
\label{sec:org2e59d33}
\begin{enumerate}
\item Initial Analysis
\label{sec:orgc1dfe73}
\begin{center}
\includegraphics[width=.9\linewidth]{/Users/tshanno/Library/Mobile Documents/com~apple~CloudDocs/zFiled By Folder/Shannon Data/2019-06-11/Rhod2-LineScan_CTRL_061119_6.png}
\end{center}
\begin{enumerate}
\item \href{file:///Users/tshanno/Library/Mobile Documents/com\~apple\~CloudDocs/zFiled By Folder/Shannon Data/2019-06-11/Rhod2-LineScan\_CTRL\_061119\_6.pzf}{Rhod2-LineScan\(_{\text{CTRL}}\)\(_{\text{061119}}\)\(_{\text{6.pzf}}\)}
\label{sec:org6fe6f28}
\begin{center}
\includegraphics[width=.9\linewidth]{/Users/tshanno/Library/Mobile Documents/com~apple~CloudDocs/zFiled By Folder/Shannon Data/2019-06-11/Analysis Rhod2-LineScan_CTRL_061119_6.png}
\end{center}
\end{enumerate}
\item Reanalysis
\label{sec:org8bf45ea}
\begin{enumerate}
\item The reanalysis of this data using the moving average still showed no aritfacts due to analysis.
\label{sec:org3f1ff53}
\end{enumerate}
\end{enumerate}
\item 2019-06-11-07
\label{sec:org94acda0}
\begin{enumerate}
\item Initial Analysis
\label{sec:org97c848b}
\begin{center}
\includegraphics[width=.9\linewidth]{/Users/tshanno/Library/Mobile Documents/com~apple~CloudDocs/zFiled By Folder/Shannon Data/2019-06-11/Rhod2-LineScan_CTRL_061119_7.png}
\end{center}
\begin{enumerate}
\item \href{file:///Users/tshanno/Library/Mobile Documents/com\~apple\~CloudDocs/zFiled By Folder/Shannon Data/2019-06-11/Rhod2-LineScan\_CTRL\_061119\_7.pzf}{Rhod2-LineScan\(_{\text{CTRL}}\)\(_{\text{061119}}\)\(_{\text{7.pzf}}\)}
\label{sec:orgff4950a}
\begin{center}
\includegraphics[width=.9\linewidth]{/Users/tshanno/Library/Mobile Documents/com~apple~CloudDocs/zFiled By Folder/Shannon Data/2019-06-11/Analysis CTRL_061119_7.png}
\end{center}
\end{enumerate}
\item Reanalysis
\label{sec:org76c218b}
\begin{enumerate}
\item The reanalysis of this data using the moving average still showed no aritfacts due to analysis.
\label{sec:orgb513d6f}
\end{enumerate}
\end{enumerate}
\item 2019-06-11-08
\label{sec:org0d148c4}
\begin{enumerate}
\item Initial Analysis
\label{sec:orge2ab84b}
\begin{center}
\includegraphics[width=.9\linewidth]{/Users/tshanno/Library/Mobile Documents/com~apple~CloudDocs/zFiled By Folder/Shannon Data/2019-06-11/Rhod2-LineScan_CTRL_061119_8.png}
\end{center}
\begin{enumerate}
\item \href{file:///Users/tshanno/Library/Mobile Documents/com\~apple\~CloudDocs/zFiled By Folder/Shannon Data/2019-06-11/Rhod2-LineScan\_CTRL\_061119\_8.pzf}{Rhod2-LineScan\(_{\text{CTRL}}\)\(_{\text{061119}}\)\(_{\text{8.pzf}}\)}
\label{sec:org1cb7128}
\begin{center}
\includegraphics[width=.9\linewidth]{/Users/tshanno/Library/Mobile Documents/com~apple~CloudDocs/zFiled By Folder/Shannon Data/2019-06-11/Analysis 061119_8.png}
\end{center}
\end{enumerate}
\item Reanalysis
\label{sec:orgc5abaa2}
\begin{enumerate}
\item The reanalysis of this data using the moving average still showed no aritfacts due to analysis.
\label{sec:orgd1452d6}
\end{enumerate}
\end{enumerate}
\item 2019-06-11-09
\label{sec:org4de47eb}
\begin{enumerate}
\item Initial Analysis
\label{sec:orge1cfe7c}
\begin{enumerate}
\item Very weird transients in band 3.
\label{sec:orgceedc95}
\end{enumerate}
\item Reanalysis
\label{sec:org9a2d21d}
\begin{enumerate}
\item The reanalysis of this data using the moving average still showed no aritfacts due to analysis.
\label{sec:org2a86747}
\end{enumerate}
\end{enumerate}
\item 2019-06-11-10
\label{sec:orgc78bb57}
\begin{enumerate}
\item Initial Analysis
\label{sec:orgfb7cd84}
\begin{enumerate}
\item Big drop in baseline after fourth beat for some reason.
\label{sec:orgd72f3bb}
\begin{center}
\includegraphics[width=.9\linewidth]{/Users/tshanno/Library/Mobile Documents/com~apple~CloudDocs/zFiled By Folder/Shannon Data/2019-06-11/Rhod2-LineScan_CTRL_061119_10.png}
\end{center}
\begin{enumerate}
\item \href{file:///Users/tshanno/Library/Mobile Documents/com\~apple\~CloudDocs/zFiled By Folder/Shannon Data/2019-06-11/Rhod2-LineScan\_CTRL\_061119\_10.pzf}{Rhod2-LineScan\(_{\text{CTRL}}\)\(_{\text{061119}}\)\(_{\text{10.pzf}}\)}
\label{sec:org8c3920f}
\begin{center}
\includegraphics[width=.9\linewidth]{/Users/tshanno/Library/Mobile Documents/com~apple~CloudDocs/zFiled By Folder/Shannon Data/2019-06-11/CRTL_061119_10.png}
\end{center}
\end{enumerate}
\end{enumerate}
\item Reanalysis
\label{sec:org20cb0ad}
\begin{enumerate}
\item The reanalysis of this data using the moving average still showed no aritfacts due to analysis.
\label{sec:orgea255bd}
\end{enumerate}
\end{enumerate}
\item 2019-06-11-11
\label{sec:org82d7336}
\begin{enumerate}
\item Initial Analysis
\label{sec:org240610e}
\begin{center}
\includegraphics[width=.9\linewidth]{/Users/tshanno/Library/Mobile Documents/com~apple~CloudDocs/zFiled By Folder/Shannon Data/2019-06-11/Rhod2-LineScan_CTRL_061119_11.png}
\end{center}
\begin{enumerate}
\item \href{file:///Users/tshanno/Library/Mobile Documents/com\~apple\~CloudDocs/zFiled By Folder/Shannon Data/2019-06-11/Rhod2-LineScan\_CTRL\_061119\_11.pzf}{Rhod2-LineScan\(_{\text{CTRL}}\)\(_{\text{061119}}\)\(_{\text{11.pzf}}\)}
\label{sec:orgbce48e7}
\begin{center}
\includegraphics[width=.9\linewidth]{/Users/tshanno/Library/Mobile Documents/com~apple~CloudDocs/zFiled By Folder/Shannon Data/2019-06-11/Analysis CTRL_061119_11.png}
\end{center}
\end{enumerate}
\item Reanalysis
\label{sec:org392afe9}
\begin{enumerate}
\item The reanalysis of this data using the moving average still showed no aritfacts due to analysis.
\label{sec:org0ce1e97}
\end{enumerate}
\end{enumerate}
\end{enumerate}
\item 2019-06-12
\label{sec:orga6a4833}
\begin{enumerate}
\item 2019-06-12-01
\label{sec:org623fa4d}
\begin{enumerate}
\item Initial analysis
\label{sec:org565827b}
\begin{center}
\includegraphics[width=.9\linewidth]{/Users/tshanno/Library/Mobile Documents/com~apple~CloudDocs/zFiled By Folder/Shannon Data/2019-06-12/Plot of Rhod2-LineScan_ISO_061219_1.png}
\end{center}
\begin{enumerate}
\item \href{file:///Users/tshanno/Library/Mobile Documents/com\~apple\~CloudDocs/zFiled By Folder/Shannon Data/2019-06-12/Rhod2-LineScan\_ISO\_061219\_1.pzf}{Rhod2-LineScan\(_{\text{ISO}}\)\(_{\text{061219}}\)\(_{\text{1.pzf}}\)}
\label{sec:org3518fc1}
\begin{center}
\includegraphics[width=.9\linewidth]{/Users/tshanno/Library/Mobile Documents/com~apple~CloudDocs/zFiled By Folder/Shannon Data/2019-06-12/Analysis ISO_061219_1.png}
\end{center}
\end{enumerate}
\item Reanalysis 2019-06-18
\label{sec:org742a957}
\begin{center}
\includegraphics[width=.9\linewidth]{/Users/tshanno/Library/Mobile Documents/com~apple~CloudDocs/zFiled By Folder/Shannon Data/Reanalysis 2019-06-18/2019-06-12-01/Analysis moving 10 micron average 2019-06-18/Rhod2-LineScan_ISO_061219_1.lsm_TimeVsTransientPeak - trimmed.png}
\end{center}
\begin{center}
\includegraphics[width=.9\linewidth]{/Users/tshanno/Library/Mobile Documents/com~apple~CloudDocs/zFiled By Folder/Shannon Data/Reanalysis 2019-06-18/2019-06-12-01/Analysis moving 10 micron average 2019-06-18/Rhod2-LineScan_ISO_061219_1.lsm_TimeVsNormalizedTransientPeak - trimmed.png}
\end{center}
\begin{center}
\includegraphics[width=.9\linewidth]{/Users/tshanno/Library/Mobile Documents/com~apple~CloudDocs/zFiled By Folder/Shannon Data/Reanalysis 2019-06-18/2019-06-12-01/Analysis moving 10 micron average 2019-06-18/Rhod2-LineScan_ISO_061219_1.lsm_BandVsNormalizedTransientPeak - trimmed.png}
\end{center}
\begin{enumerate}
\item Transient peak data shows a gradient which is uncovered after normalization to initial level.  Band data seem to indicate the initial increase is taking place pretty far up the cell in the middle.
\label{sec:orgd7039f7}
\begin{center}
\includegraphics[width=.9\linewidth]{/Users/tshanno/Library/Mobile Documents/com~apple~CloudDocs/zFiled By Folder/Shannon Data/Reanalysis 2019-06-18/2019-06-12-01/Analysis moving 25 micron average 2019-06-18/Rhod2-LineScan_ISO_061219_1.lsm_TimeVsNormalizedTimeToHalf - trimmed.png}
\end{center}
\begin{center}
\includegraphics[width=.9\linewidth]{/Users/tshanno/Library/Mobile Documents/com~apple~CloudDocs/zFiled By Folder/Shannon Data/Reanalysis 2019-06-18/2019-06-12-01/Analysis moving 25 micron average 2019-06-18/Rhod2-LineScan_ISO_061219_1.lsm_TimeVsNormalizedTimeTo90.png}
\end{center}
\begin{center}
\includegraphics[width=.9\linewidth]{/Users/tshanno/Library/Mobile Documents/com~apple~CloudDocs/zFiled By Folder/Shannon Data/Reanalysis 2019-06-18/2019-06-12-01/Analysis moving 25 micron average 2019-06-18/Rhod2-LineScan_ISO_061219_1.lsm_BandVsNormalizedTimeTo90 - trimmed.png}
\end{center}
\item Decline data is showing a gradient but only at the very top of the cell.  He hit this one high.
\label{sec:org30c85af}
\end{enumerate}
\end{enumerate}
\item 2019-06-12-03
\label{sec:org08db18c}
\begin{enumerate}
\item Initial Analysis
\label{sec:org53f8ea7}
\begin{center}
\includegraphics[width=.9\linewidth]{/Users/tshanno/Library/Mobile Documents/com~apple~CloudDocs/zFiled By Folder/Shannon Data/2019-06-12/Rhod2-LineScan_ISO_061219_3.png}
\end{center}
\href{file:///Users/tshanno/Library/Mobile Documents/com\~apple\~CloudDocs/zFiled By Folder/Shannon Data/2019-06-12/Rhod2-LineScan\_ISO\_061219\_3.pzf}{Rhod2-LineScan\(_{\text{ISO}}\)\(_{\text{061219}}\)\(_{\text{3.pzf}}\)}
\begin{center}
\includegraphics[width=.9\linewidth]{/Users/tshanno/Library/Mobile Documents/com~apple~CloudDocs/zFiled By Folder/Shannon Data/2019-06-12/Analysis ISO_061219_3.png}
\end{center}
\item Reanalysis 2019-06-17
\label{sec:org0697337}
\begin{center}
\includegraphics[width=.9\linewidth]{/Users/tshanno/Library/Mobile Documents/com~apple~CloudDocs/zFiled By Folder/Shannon Data/Reanalysis 2019-06-17/Cell 3/Analysis 10 micon moving average 2019-06-17/Rhod2-LineScan_ISO_061219_3.lsm_TimeVsTransientPeak - trimmed.png}
\end{center}
\begin{center}
\includegraphics[width=.9\linewidth]{/Users/tshanno/Library/Mobile Documents/com~apple~CloudDocs/zFiled By Folder/Shannon Data/Reanalysis 2019-06-17/Cell 3/Analysis 10 micon moving average 2019-06-17/Rhod2-LineScan_ISO_061219_3.lsm_TimeVsTransientPeak - trimmed.png}
\end{center}
\begin{center}
\includegraphics[width=.9\linewidth]{/Users/tshanno/Library/Mobile Documents/com~apple~CloudDocs/zFiled By Folder/Shannon Data/Reanalysis 2019-06-17/Cell 3/Analysis 25 micon moving average 2019-06-17/Rhod2-LineScan_ISO_061219_3.lsm_TimeVsTimeTo90 - trimmed.png}
\end{center}
\begin{center}
\includegraphics[width=.9\linewidth]{/Users/tshanno/Library/Mobile Documents/com~apple~CloudDocs/zFiled By Folder/Shannon Data/Reanalysis 2019-06-17/Cell 3/Analysis 25 micon moving average 2019-06-17/Rhod2-LineScan_ISO_061219_3.lsm_TimeVsTimeToHalf.png}
\end{center}
\begin{enumerate}
\item There may be a delay in the transient peak data.  The rest will need further analysis.  Probably needs to be normalized to initial level.
\label{sec:org13c5265}
\end{enumerate}
\end{enumerate}
\item 2019-06-12-04
\label{sec:org1ad5a0b}
\begin{enumerate}
\item Initial Analysis
\label{sec:org578308d}
\begin{center}
\includegraphics[width=.9\linewidth]{/Users/tshanno/Library/Mobile Documents/com~apple~CloudDocs/zFiled By Folder/Shannon Data/2019-06-12/Rhod2-LineScan_ISO_061219_4.png}
\end{center}
\begin{enumerate}
\item \href{file:///Users/tshanno/Library/Mobile Documents/com\~apple\~CloudDocs/zFiled By Folder/Shannon Data/2019-06-12/Rhod2-LineScan\_ISO\_061219\_4.pzf}{Rhod2-LineScan\(_{\text{ISO}}\)\(_{\text{061219}}\)\(_{\text{4.pzf}}\)}
\label{sec:orgf1a088b}
\begin{center}
\includegraphics[width=.9\linewidth]{/Users/tshanno/Library/Mobile Documents/com~apple~CloudDocs/zFiled By Folder/Shannon Data/2019-06-12/Analysis ISO_061219_4.png}
\end{center}
\end{enumerate}
\item Reanalysis 2019-06-17
\label{sec:org209c7b6}
\begin{center}
\includegraphics[width=.9\linewidth]{/Users/tshanno/Library/Mobile Documents/com~apple~CloudDocs/zFiled By Folder/Shannon Data/Reanalysis 2019-06-17/Cell 4/Analysis 10 micron moving average 2019-06-17/Rhod2-LineScan_ISO_061219_4.lsm_TimeVsTransientPeak.png}
\end{center}
\begin{center}
\includegraphics[width=.9\linewidth]{/Users/tshanno/Library/Mobile Documents/com~apple~CloudDocs/zFiled By Folder/Shannon Data/Reanalysis 2019-06-17/Cell 4/Analysis 25 micron moving average 2019-06-17/Rhod2-LineScan_ISO_061219_4.lsm_TimeVsTimeTo90.png}
\end{center}
\begin{center}
\includegraphics[width=.9\linewidth]{/Users/tshanno/Library/Mobile Documents/com~apple~CloudDocs/zFiled By Folder/Shannon Data/Reanalysis 2019-06-17/Cell 4/Analysis 25 micron moving average 2019-06-17/Rhod2-LineScan_ISO_061219_4.lsm_TimeVsTimeTo90 - trimmed.png}
\end{center}
\begin{center}
\includegraphics[width=.9\linewidth]{/Users/tshanno/Library/Mobile Documents/com~apple~CloudDocs/zFiled By Folder/Shannon Data/Reanalysis 2019-06-17/Cell 4/Analysis 25 micron moving average 2019-06-17/Rhod2-LineScan_ISO_061219_4.lsm_TimeVsTimeToHalf - trimmed.png}
\end{center}
\begin{center}
\includegraphics[width=.9\linewidth]{/Users/tshanno/Library/Mobile Documents/com~apple~CloudDocs/zFiled By Folder/Shannon Data/Reanalysis 2019-06-17/Cell 4/Analysis 10 micron moving average 2019-06-17/Rhod2-LineScan_ISO_061219_4.lsm_TimeVsTransientPeak.png}
\end{center}
\begin{enumerate}
\item No surprise, this analysis confirms what we saw with the initial band analysis.  Both the time to 90\% and the time to half decline is delayed further up the cell.  The increased detail in this data should help me to make better calculations.
\label{sec:org8cbc91b}
\item The 10 μm band gives OK results for the peak - which also shows a gradient.  I felt the 25 μm band gave better results for the declines.
\label{sec:org4e3a7f4}
\item Judging from the transient peak spatial data it looks like the ISO hit about half way up the cell.  This needs to be looked at further.
\label{sec:org56883d7}
\end{enumerate}
\end{enumerate}
\end{enumerate}
\item 2019-06-19
\label{sec:org7cf2d01}
\begin{enumerate}
\item \textbf{DAN'S NOTES}
\label{sec:org9b984bc}
\textbf{Best cells}

\textbf{\emph{Rank Order -First to Last;}} Analysis Priority

Cells 21, 14, 12, 19, 25, 18, 9

\emph{\textbf{Irregular Effects}}

Cells 3, 4,5 7,70,11,17,23,24

\emph{\textbf{Waves}}

Cells 1,2,6,8,13,15,16,22

\emph{\textbf{No effect}}

Cell 20
\item 2019-06-19-09
\label{sec:orgfed178c}
\begin{center}
\includegraphics[width=.9\linewidth]{/Users/tshanno/Library/Mobile Documents/com~apple~CloudDocs/zFiled By Folder/Shannon Data/Analysis 2019-06-22 23 and 24 of 2019-06-19 data/2019-06-19-09/2019-06-19-09.png}
\end{center}
\begin{enumerate}
\item The bands near the top of the cell away from the pipette are very weird looking.  A lot of the bands have a little hitch in the decline but its exaggerated on that end relative to the peak of the transient.  Movement artifact?  Those bands will likely have to be cut out of the analysis.
\label{sec:org51c7be1}
\begin{center}
\includegraphics[width=.9\linewidth]{/Users/tshanno/Library/Mobile Documents/com~apple~CloudDocs/zFiled By Folder/Shannon Data/Analysis 2019-06-22 23 and 24 of 2019-06-19 data/2019-06-19-09/Rhod2-LineScan_ISO_F_062019_9.lsm_TimeVsNormalizedTransientPeak - trimmed.png}
\end{center}
\item Looks like there's a moving baseline here but there's definitely a gradient.  I can probably find a better way to analyze this to make it usable.  This one is 10 μm bands.
\label{sec:orgdc0c686}
\end{enumerate}
\item 2019-06-19-12
\label{sec:orgded09a6}
\begin{center}
\includegraphics[width=.9\linewidth]{/Users/tshanno/Library/Mobile Documents/com~apple~CloudDocs/zFiled By Folder/Shannon Data/Analysis 2019-06-22 23 and 24 of 2019-06-19 data/2019-06-19-12/2019-06-19-21.png}
\end{center}
\begin{enumerate}
\item Definitely a gradient in time to 90\% but not time to 50\% (25 micron band)
\label{sec:org5ff1d53}
\begin{center}
\includegraphics[width=.9\linewidth]{/Users/tshanno/Library/Mobile Documents/com~apple~CloudDocs/zFiled By Folder/Shannon Data/Analysis 2019-06-22 23 and 24 of 2019-06-19 data/2019-06-19-12/Rhod2-LineScan_ISO_F_062019_12.lsm_TimeVsNormalizedTimeToHalf - filtered.png}
\end{center}
\begin{center}
\includegraphics[width=.9\linewidth]{/Users/tshanno/Library/Mobile Documents/com~apple~CloudDocs/zFiled By Folder/Shannon Data/Analysis 2019-06-22 23 and 24 of 2019-06-19 data/2019-06-19-12/Rhod2-LineScan_ISO_F_062019_12.lsm_TimeVsNormalizedTimeTo90 - filetered.png}
\end{center}
\item Gradient in time to peak, which definitely comes earlier than time to 90\% but there's a "pause" in the middle.  I'm not sure what that is. (10 micron band)
\label{sec:org9aad718}
\begin{center}
\includegraphics[width=.9\linewidth]{/Users/tshanno/Library/Mobile Documents/com~apple~CloudDocs/zFiled By Folder/Shannon Data/Analysis 2019-06-22 23 and 24 of 2019-06-19 data/2019-06-19-12/Rhod2-LineScan_ISO_F_062019_12.lsm_TimeVsNormalizedTransientPeak - trimmed.png}
\end{center}
\end{enumerate}
\item 2019-06-19-14
\label{sec:orge663116}
\begin{center}
\includegraphics[width=.9\linewidth]{/Users/tshanno/Library/Mobile Documents/com~apple~CloudDocs/zFiled By Folder/Shannon Data/Analysis 2019-06-22 23 and 24 of 2019-06-19 data/2019-06-19-14/2019-06-19-14.png}
\end{center}
\begin{center}
\includegraphics[width=.9\linewidth]{/Users/tshanno/Library/Mobile Documents/com~apple~CloudDocs/zFiled By Folder/Shannon Data/Analysis 2019-06-22 23 and 24 of 2019-06-19 data/2019-06-19-14/Rhod2-LineScan_ISO_F_062019_14.lsm_TimeVsNormalizedTransientPeak - trimmed.png}
\end{center}
\begin{center}
\includegraphics[width=.9\linewidth]{/Users/tshanno/Library/Mobile Documents/com~apple~CloudDocs/zFiled By Folder/Shannon Data/Analysis 2019-06-22 23 and 24 of 2019-06-19 data/2019-06-19-14/Rhod2-LineScan_ISO_F_062019_14.lsm_TimeVsNormalizedTimeTo90 - trimmed.png}
\end{center}
\begin{center}
\includegraphics[width=.9\linewidth]{/Users/tshanno/Library/Mobile Documents/com~apple~CloudDocs/zFiled By Folder/Shannon Data/Analysis 2019-06-22 23 and 24 of 2019-06-19 data/2019-06-19-14/Rhod2-LineScan_ISO_F_062019_14.lsm_TimeVsNormalizedTimeToHalf - trimmed.png}
\end{center}
\begin{enumerate}
\item Really the same as 12 in almost all ways.
\label{sec:orgee853d8}
\end{enumerate}

\item 2019-06-18-18
\label{sec:org5fab30f}
\begin{center}
\includegraphics[width=.9\linewidth]{/Users/tshanno/Library/Mobile Documents/com~apple~CloudDocs/zFiled By Folder/Shannon Data/Analysis 2019-06-22 23 and 24 of 2019-06-19 data/2019-06-19-18/2019-06-19-18.png}
\end{center}
\begin{center}
\includegraphics[width=.9\linewidth]{/Users/tshanno/Library/Mobile Documents/com~apple~CloudDocs/zFiled By Folder/Shannon Data/Analysis 2019-06-22 23 and 24 of 2019-06-19 data/2019-06-19-18/Rhod2-LineScan_ISO_F_062019_18.lsm_TimeVsNormalizedTimeTo90 - trimmed.png}
\end{center}
\begin{center}
\includegraphics[width=.9\linewidth]{/Users/tshanno/Library/Mobile Documents/com~apple~CloudDocs/zFiled By Folder/Shannon Data/Analysis 2019-06-22 23 and 24 of 2019-06-19 data/2019-06-19-18/Rhod2-LineScan_ISO_F_062019_18.lsm_TimeVsNormalizedTransientPeak.png}
\end{center}
\begin{enumerate}
\item Just like 12
\label{sec:orgb8b0765}
\end{enumerate}
\item 2019-06-19-19
\label{sec:orge5bc484}
\begin{center}
\includegraphics[width=.9\linewidth]{/Users/tshanno/Library/Mobile Documents/com~apple~CloudDocs/zFiled By Folder/Shannon Data/Analysis 2019-06-22 23 and 24 of 2019-06-19 data/2019-06-19-19/2019-06-19-19.png}
\end{center}
\begin{center}
\includegraphics[width=.9\linewidth]{/Users/tshanno/Library/Mobile Documents/com~apple~CloudDocs/zFiled By Folder/Shannon Data/Analysis 2019-06-22 23 and 24 of 2019-06-19 data/2019-06-19-19/Rhod2-LineScan_ISO_F_062019_19.lsm_TimeVsNormalizedTimeTo90 - trimmed.png}
\end{center}
\begin{center}
\includegraphics[width=.9\linewidth]{/Users/tshanno/Library/Mobile Documents/com~apple~CloudDocs/zFiled By Folder/Shannon Data/Analysis 2019-06-22 23 and 24 of 2019-06-19 data/2019-06-19-19/Rhod2-LineScan_ISO_F_062019_19.lsm_TimeVsNormalizedTimeToHalf - trimmed.png}
\end{center}
\begin{center}
\includegraphics[width=.9\linewidth]{/Users/tshanno/Library/Mobile Documents/com~apple~CloudDocs/zFiled By Folder/Shannon Data/Analysis 2019-06-22 23 and 24 of 2019-06-19 data/2019-06-19-19/Rhod2-LineScan_ISO_F_062019_19.lsm_TimeVsNormalizedTransientPeak - trimmed.png}
\end{center}
\begin{enumerate}
\item This one showed a change in both the time to 90\% and the time to 50\%.  All of these are 25 micron band analysis.  I think I should probably just use This from now on.  It looks better.
\label{sec:orge5cdcd6}
\end{enumerate}
\item 2019-06-19-21
\label{sec:org0ad15fa}
\begin{center}
\includegraphics[width=.9\linewidth]{/Users/tshanno/Library/Mobile Documents/com~apple~CloudDocs/zFiled By Folder/Shannon Data/Analysis 2019-06-22 23 and 24 of 2019-06-19 data/2019-06-19-21/2019-06-19-21.png}
\end{center}
\begin{center}
\includegraphics[width=.9\linewidth]{/Users/tshanno/Library/Mobile Documents/com~apple~CloudDocs/zFiled By Folder/Shannon Data/Analysis 2019-06-22 23 and 24 of 2019-06-19 data/2019-06-19-21/Rhod2-LineScan_ISO_F_062019_21.lsm_TimeVsNormalizedTransientPeak.png}
\end{center}
\begin{center}
\includegraphics[width=.9\linewidth]{/Users/tshanno/Library/Mobile Documents/com~apple~CloudDocs/zFiled By Folder/Shannon Data/Analysis 2019-06-22 23 and 24 of 2019-06-19 data/2019-06-19-21/Rhod2-LineScan_ISO_F_062019_21.lsm_TimeVsNormalizedTransientPeak - filtered.png}
\end{center}
\begin{center}
\includegraphics[width=.9\linewidth]{/Users/tshanno/Library/Mobile Documents/com~apple~CloudDocs/zFiled By Folder/Shannon Data/Analysis 2019-06-22 23 and 24 of 2019-06-19 data/2019-06-19-21/Rhod2-LineScan_ISO_F_062019_21.lsm_TimeVsNormalizedTimeToPeak - filtered.png}
\end{center}
\begin{center}
\includegraphics[width=.9\linewidth]{/Users/tshanno/Library/Mobile Documents/com~apple~CloudDocs/zFiled By Folder/Shannon Data/Analysis 2019-06-22 23 and 24 of 2019-06-19 data/2019-06-19-21/Rhod2-LineScan_ISO_F_062019_21.lsm_TimeVsNormalizedTimeToHalf.png}
\end{center}
\begin{center}
\includegraphics[width=.9\linewidth]{/Users/tshanno/Library/Mobile Documents/com~apple~CloudDocs/zFiled By Folder/Shannon Data/Analysis 2019-06-22 23 and 24 of 2019-06-19 data/2019-06-19-21/Rhod2-LineScan_ISO_F_062019_21.lsm_TimeVsNormalizedTimeTo90.png}
\end{center}
\begin{enumerate}
\item The time to 90\% and 50\% decline are almost perfect
\label{sec:orgf7422dc}
\item The time to peak showed usable data.  Surprisingly This increased closest to the pipette not decreased
\label{sec:org2faeb0c}
\item The transient peak data looks good but I had to cut a lot of bands on the ISO edge of the myocyte.  These increased, then decreased in an exaggerated way.  The figues above are with and without filtering.
\label{sec:orga7320e6}
\end{enumerate}
\item 2019-06-19-25
\label{sec:orgea4e17d}
\begin{center}
\includegraphics[width=.9\linewidth]{/Users/tshanno/Library/Mobile Documents/com~apple~CloudDocs/zFiled By Folder/Shannon Data/Analysis 2019-06-22 23 and 24 of 2019-06-19 data/2019-06-19-25/2019-06-19-25.png}
\end{center}
\begin{center}
\includegraphics[width=.9\linewidth]{/Users/tshanno/Library/Mobile Documents/com~apple~CloudDocs/zFiled By Folder/Shannon Data/Analysis 2019-06-22 23 and 24 of 2019-06-19 data/2019-06-19-25/Rhod2-LineScan_ISO_F_062019_25.lsm_TimeVsNormalizedTimeTo90 - trimmed.png}
\end{center}
\begin{center}
\includegraphics[width=.9\linewidth]{/Users/tshanno/Library/Mobile Documents/com~apple~CloudDocs/zFiled By Folder/Shannon Data/Analysis 2019-06-22 23 and 24 of 2019-06-19 data/2019-06-19-25/Rhod2-LineScan_ISO_F_062019_25.lsm_TimeVsNormalizedTimeToHalf - trimmed.png}
\end{center}
\begin{center}
\includegraphics[width=.9\linewidth]{/Users/tshanno/Library/Mobile Documents/com~apple~CloudDocs/zFiled By Folder/Shannon Data/Analysis 2019-06-22 23 and 24 of 2019-06-19 data/2019-06-19-25/Rhod2-LineScan_ISO_F_062019_25.lsm_TimeVsNormalizedTransientPeak.png}
\end{center}
\begin{enumerate}
\item This data is really noisy.  Maybe the transient peak data is usable.
\label{sec:org9c6caf2}
\end{enumerate}
\end{enumerate}

\item 2019-06-10 Analysis
\label{sec:orgb8fb82a}
\begin{itemize}
\item Note taken on \textit{[2019-06-10 Mon 10:29] } \\
Analyzed 2019-05-31 ro 2019-06-07.  I'm concerned about the analysis.

It looks to me like the "best" cells from Dan showed a difference in the time to 90\% decline.  These were 2019-06-07-2 and 2019-06-07-3.  These were collected after Dan uped the ISO concentrtion to 2 uM.

What is concerning is that band 3 shows a linear decline in these cells.  This might be due to a gradual drift in the analysis due to the fact that the stimulation rate wasn't exactly 0.5 Hz.  The "time to 90\%" may have actually been the time to the end of the segment that I was analyzing.  This should, of course have been a window where the fill transient could be visualized.  However, it was obvious that it was cutting off part way down the transient late in the analysis at the later time points.

\item The other thing that worries me is that I think we need control data to make sure that even the data that we didn't think came from "good" cells showed a decline in both time to peak and time to 90\%.

\item I think we should keep the ISO at 2 uM

\item \textbf{I think we need control data to make sure that what I'm seeing isn't an artifact.}
\end{itemize}

\item 2019-06-11 Reanalysis
\label{sec:org9c31cf3}
\begin{itemize}
\item Note taken on \textit{[2019-06-11 Tue 08:12] } \\
Reanalyzed 2019-06-07-2 and 2019-06-07-3.

Reanalysis 1 was simply playing with the begin and end transient times to center the transients since we weren't at exactly 0.5 Hz.

Reanalysis 2 was reanalysis 1 with a 9 point box car smooth.

Both sets of data look better.

Still no differences in either cell in the time to half max.

Cell 2 seems to definitely have a slower response in band 3 than bands 1 and 2 in the time to 90\% decline.  This confirms what the first analysis showed but the data look better and less like it might be artifact.

Interestingly, cell 3 had no shortening at all in band 3 for time to 90\% decline.  And there appears to be a real difference between bands 1 and 2 (unlike cell 2).

Reanalysis 2 has all of the parameters hard coded in so I'll know exactly what I used.
\end{itemize}

\item 2019-06-15 Analysis of 2019-06-11 and 2019-06-12
\label{sec:orgc0c29b6}
\begin{itemize}
\item Note taken on \textit{[2019-06-16 Sun 09:17] } \\
2019-06-11 was all control data with no ISO perfusion.  I did some analysis of the 2019-06-11 data but didn't finish.  The analysis of the cells up to number 10 was wrong because I wasn't using the right spatial parameters.  I'll have to redo cells 5-9.

Cells 10 and 11 were done correctly and showed no decline in the time to half decline or time to 90\% decline.  So I' reasonably confident that the analysis isn't adding any artifacts.

2019-06-12 was all ISO data.  All bands in Cells 1,3 and 4 declined but cells 1 and 3 didn't show any apparent difference in the decline of the times to half and 90\%.

Only cell 4 showed the response we are looking 4.  Band 3 definitely responded late and possibly more slowly to the ISO for both the time to 50\% decline and time to 90\% decline.

Interpretation:

\begin{itemize}
\item The mathematical analysis isn't adding an artifact.

\item I'm a bit concerned about the "time to declines" that I'm getting in terms of the numbers.  Given that the whole transient should be only 2000 ms long, a time to 90\% decline of 3000 seems wrong.  I'll have to take a further look at the code to see where this discrepancy is coming from.

\item I think Dan might be hitting too much of the cell on some of these so that the ISO effect disappears.  Given the amount of trouble he's having controlling this, I may have to do some further analysis in order to make this data quantitative.
\end{itemize}

I'm thinking of keeping the 50 pixel banding but moving down the cell line by line (average of 25 pixels on each side).  I'll look at each line and determine when they hit some mark, let's say a time to 25\% decline in time to 90\%.  I'll determine more or less where along the length of the cell the first delay in this time seems to be and figure that's where the ISO stops hitting the cell.  From there, determining the rate at which the response propagates up the cell whouldn't be hard.
\end{itemize}
\item 2019-06-17 Re-analysis of 2019-06-12
\label{sec:org73bc3ee}
\begin{itemize}
\item Note taken on \textit{[2019-06-18 Tue 07:28] } \\
Only analyzed cells 3 and 4.

This revised analysis is a moving spatial average of the flursesence.  So, for instance, the first "band" of the 25 μm is actually centered on a line 12.5 μm from the edge with a total width of 25.  The analysis then moves 1 pixel over and repeats for the next band.  This gives filtered data over the entire length of the cell.

I also generated figures which have band number on the x axis.  In this case, each line represents a transient.  My hope is that I can better localize exactly where the ISO is hitting the cell by looking at which band immediately respond to the ISO (those that are being perfused) and which are delayed and by how much (the longer the delay, the further from the region being perfused.

This led to some interesting results.

It looks like the ISO hit cell 4 about half way up.  This needs to be looked at further.  The cell responded well and You can definitely see a spatial gradient in the response in terms of time to half decline, time to 90\% decline and transient peak.

There might be something there in cell 3.  The transient peak data seems to show a gradient but its hard to tell.  This data really needs to be normalized to the intial level.
\end{itemize}

Note also that I fixed the time bug.  The times to half and 90\% decline should be accurate now.
\item 2019-06-18 Reanalysis of 2019-06-12-01, 2019-06-11-09 to 11 and 2019-06-07
\label{sec:org16bd6b2}
\begin{itemize}
\item Note taken on \textit{[2019-06-18 Tue 14:36] } \\
I also reanalyzed 2019-06-12-03 and 2019-06-12-04 so that the normalized graphs would be generated.  Should make it easier to evaluate this data, especially 03.
\item Note taken on \textit{[2019-06-18 Tue 08:37] } \\
Added a normalization protocol to the analysis so that now we have graphs of data normalized to the initial levels.

2019-06-12-01
This cell actually did respond to ISO but the transient data kind of looks like it responded in the middle and them the response diffused out to the ends.  The time to declines who a gradient only on the top end.  Note the band data where the final decrease in time to 90\% takes place very lat at the very top 20\% or so.  Dan hit this one really high be there might be something we can get out of it.

2019-06-11-09 to 11
These were the only control cells that I analyzed.  I'll go back and reanalyze later.  Suffice it to say that there was no apparent change due to the analysis and there do not appear to be any artifacts added on that account.

2019-06-07-4
There definitely is a gradient here in the time to 90\% decline.  There's nothing there in the time to half decline.  The gradient is there in the peak transients as well.

Both seem to show that the response basically stops and doesn't make it all the way up the cell.

2019-06-07-3

Note that the profile for this cell went all the way to the edge.  Used 500.

There's some unsteadiness in the peak transient data over time but it looks like there's a gradient there.   When You look at the raw transient traces over time it does look like the peaks go up and down a bit.  May not be able to use this.

The band Vs. tt90 data seems to show that this cell got hit about halfway up.  Definitely a gradient in this data but its very noisy.  There might be something in the tt50 but its too noisy to tease out.

2019-06-07-2

There might be a gradient in the tt90 data but I wouldn't bet that it will amount to anything.   Nothing in the tt50 data.

The transient peak data shows a definite gradient.  The 25 um band data looks better.
\end{itemize}
\item 2019-06-19 Reanalysis of 2019-06-11-05 to 08
\label{sec:org81584e7}
\begin{itemize}
\item Note taken on \textit{[2019-06-19 Wed 13:48] } \\
2019-06-04-1 and -2 This data's no good.  It looks like there might be a gradient in the transients but the peaks are almost random as a function of space.
\item Note taken on \textit{[2019-06-19 Wed 13:18] } \\
2019-06-06-05 Was pretty noisy.  The peak transient data migh tbe useful.
\item Note taken on \textit{[2019-06-19 Wed 07:27] } \\
Reanalyzed 2019-06-11-05, 06, 07 and 08 as on 2019-06-18 and 19.  There was, as expected, no artifacts associated with the analysis in these control cells.

I did add a line to the script to save the workspace.  This may be a good idea as having the data saved may help with the inevitable further analysis that Will need to take place with this data.
\end{itemize}
\item 2019-06-22,23 \& 24 Analysis of 2019-06-19 data
\label{sec:org66daace}
\begin{itemize}
\item Note taken on \textit{[2019-06-24 Mon 07:06] } \\
\end{itemize}
Only analyzed Dan's "best" cells.  I'm going to have to get around to analyzing soe of the more iffy cells eventually to make sure there isn't something significnt hiding in there.

Some of This data was analyzed with 10 micron moving averages for the transient peaks but 25 micron moving averages seem to be giving the best analysis and I think I'll stick to that from now on.

This was for the most part usable data.  2019-06-19-21 was, indeed, the best and This might make a good sample cell.

Generally speaking the Peak transient data is the least noisy but it often seems to go up an down and the peaks most down stream are sometime lower than the ones more toward the middle of the cell.   Its possible that Dan is hitting the cells high enough up to where the ISO is diluting out as it is washed down stream over the cell.  However Note that the transient decline data does not show This tendency towards dilution of effect down stream.

The time to half decline remains unreliable but it is definitely there on occasion.  For whatever reason it doesn't appear to be as sensitive as the time to 90\% decline.

All of the effects tend to stop short rather than propagate all the way up the cell.

\begin{enumerate}
\item Could be the SR Ca pump is so cranked up that its taking up the Ca faster than it can diffuse (both inside and outside the SR).
\item Could be that some essential element is not diffusable.  What is the role of the cytoskeleton?  Perhaps some experiments with cytochalasion?
\end{enumerate}
\end{enumerate}
\section{\textbf{Tasks}}
\label{sec:org3908e78}
\subsection{{\bfseries\sffamily TODO} \href{//\%3c20180124110537.49E71604D8E7@pmdist301.st-va.ncbi.nlm.nih.gov\%3E}{Read Wayne's new article}}
\label{sec:org34b184b}
\subsection{{\bfseries\sffamily TODO} \href{//\%3c20180124110952.02870604D8E6@pmdist301.st-va.ncbi.nlm.nih.gov\%3E}{Read Mike/Xun article}}
\label{sec:orgefe62f8}
\subsection{{\bfseries\sffamily TODO} \href{//\%3c20180125111327.E50D720155E0@esupp01.be-md.ncbi.nlm.nih.gov\%3E}{Another Fill article}}
\label{sec:orga91cf54}
\subsection{{\bfseries\sffamily DONE} Analyze Dan's latest data before phone call with Don at noon.}
\label{sec:orgb597721}
\subsubsection{Looks to me like he's overdosing with ISO.}
\label{sec:orgae6ba3b}
\subsubsection{Check solutions}
\label{sec:orgca2e53b}
\subsubsection{Decrease dose to 100 nM}
\label{sec:org65ef44e}
\subsection{{\bfseries\sffamily DONE} \href{//\%3cd86b20f74b8f4abbad2a713cd91e9e31@MLBXCH15.cs.myharris.net\%3E}{Renew IDL Maintenance by July 1}}
\label{sec:orgb168438}
\subsection{{\bfseries\sffamily DONE} TRAVEL FORMS!}
\label{sec:org28bde74}
\textit{[2018-07-17 Tue]}
\subsection{{\bfseries\sffamily DONE} \href{//\%3c959be2bf71a443ac9bbfb5eac56f4fd8@646005169\%3E}{Renew Matlab (90 day notice)}}
\label{sec:orgb0f1385}
\subsection{{\bfseries\sffamily DONE} Call Dan}
\label{sec:org85b05c3}
\subsection{{\bfseries\sffamily DONE} \href{//\%3c172ACE66-BF7C-48F6-908D-1589C1209E4B@rush.edu\%3E}{Do work for Don on AKAP stuff}}
\label{sec:org3b104eb}
\subsection{{\bfseries\sffamily TODO} \href{//\%3c172ACE66-BF7C-48F6-908D-1589C1209E4B@rush.edu\%3E}{Work on methods paper}}
\label{sec:org9e5728e}
\subsection{{\bfseries\sffamily DONE} FU: Dan (due June 12) \href{//\%3cc6ef34a6c1ed42aa9d2062694f0be929@RUPW-EXCHMAIL01.rush.edu\%3E}{Annual Renewal Notification IACUC \#17-040}}
\label{sec:org2dd1ec8}

\subsection{{\bfseries\sffamily TODO} Prepare a PowerPoint for call with Don \href{//\%3c61415DB3-D295-4CE7-BC84-5CED3BABF338@rush.edu\%3E}{Re: Could we set up a call?}}
\label{sec:org4a56131}
\end{document}